% "World of Adventure" campaign guide - Classes
% Written by Christopher Thomas.

\chapter{Classes}
\label{sect-classes}

\section{Modified Classes}

%
\subsection{Bard}

Bardic magic relies on \textit{performances}, so there are limitations with
respect to magic items:

\begin{itemize}
\item Bards can't make scrolls the way wizards can.
\item Bards \textit{can} use wizard scrolls, but this requires a Use Magic
Device check (with a slightly easier DC, per the modified skill description).
\end{itemize}
%

%
\subsection{Paladin}

The various paladin organizations have strong beliefs about how the world
should work, and their members draw power from being exemplars of those
beliefs. For campaigns that use ``alignment'', they can be any Lawful
alignment:

\begin{itemize}
\item An example of a LG paladin would be Superman when he's in full ``boy
scout'' mode.
\item An example of a LN paladin would be Judge Dredd.
\item An example of a LE paladin would be Darth Vader (original trilogy
version).
\end{itemize}

Mechanical changes are as follows:

\begin{itemize}
%
\item The ``Detect Evil'' power instead acts as the ``Know Alignment''
spell (per the ``New Spells'' section).
%
\item The ``Smite Evil'' power works against any target that is strongly
opposed to the paladin's ethos, interpreted at the DM's discretion. This
adds the paladin's class level \textbf{and} their charisma bonus to
\textbf{both} the to-hit roll \textbf{and} the damage roll. Damage reduction
is only bypassed for Immaterium-powered entities such as undead, ghosts,
posession victims, and so forth (interpreted at the DM's discretion).
%
\item Paladins can spend a feat to take the ``Armour Training'' fighter
ability (prerequisite: Paladin level 3).
%
\item ``Lay On Hands'' has floor(level/2) + cha bonus uses, and does
floor(level/2) dice of healing. As with other healing magic, it heals
(roll result) vigor points \textit{and} (number of dice) wound points.
%
\item ``Divine Health'' doesn't make you immune to disease. It gives you
a bonus of +(5 + paladin level) to saving throws to resist disease and to
recover from disease, and a bonus of +(paladin level) to saving throws to
resist or recover from poison.
%
\item ``Divine Bond'''s weapon option doesn't involve a spirit; it's an
Immaterium effect, just as with Divine spellcasting. Per the SRD's text,
the weapon enhancement from Divine Bond stacks with magical weapon
enhancement, and if the weapon already has a +1 enhancement, a special
effect can be chosen rather than needing to add an enhancement bonus first.
Stacking only works with permanent enhancement bonuses, not temporary
bonuses (such as those granted by the ``Magic Weapon'' spell).
%
\item ``Divine Bond'''s animal option functions as a druid's animal
companion, per the SRD's text. The animal companion gains intelligence by
the same method a wizard's familiar does: by using its master's mind to
think with (with the same resulting limitations). Consult the familiar table
for the base intelligence score, using (paladin level - 4) as the equivalent
wizard level. Divine Bond animals cannot teleport.
%
\end{itemize}
%

%
\subsection{Sorcerer}

Sorcerers are charisma casters, casting by \textit{intuition} rather than
using formal training, so there are limitations with respect to magic items:

\begin{itemize}
\item Sorcerers can't make scrolls the way wizards can.
\item Sorcerers \textit{can} use wizard scrolls, but this requires a Use
Magic Device check (with a slightly easier DC, per the modified skill
description).
\end{itemize}
%

%
%
\section{New Classes}

%
\subsection{Expert (Magical)}

This is a variant of the NPC ``Expert'' class that has spellcasting.

Progression and features are per the Expert class, with the following
changes:

\begin{itemize}
\item The EM gets 8 class skills rather than 10.
\item The EM gets 4+Int skill ranks per level, rather than 6 (they
additionally get 1 bonus rank replacing ``favoured class'', as described
in ``character generation'').
\item The EM gains spells as if they were a Cleric, Druid, or Wizard one
level lower than their EWM level. They do not gain any of the other class
features (at DM's discretion some of these might be bought as feats).
\item At level 1, the EM has 3 orison or cantrip spell slots.
\end{itemize}
%

Typical professions would be porters/teamsters, chefs, alchemists, and
doctors/paramedics.

In the 2022 campaign, many industries were staffed by wizards (alchemy) or
clerics (food service), and there was a huge jump in capability between
"expert with a cantrip feat" and "full wizard/cleric". This class attempts
to bridge that gap. Full clerics/druids/wizards will tend to be specialists,
with most of any given industry being magical experts instead.

%
% This is the end of the file.
