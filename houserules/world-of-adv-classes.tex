% "World of Adventure" campaign guide - Classes
% Written by Christopher Thomas.

\chapter{Classes}
\label{sect-classes}

\section{Modified Classes}

%
\subsection{Bard}

Bardic magic relies on \textit{performances}, so there are limitations with
respect to magic items:

\begin{itemize}
\item Bards can't make scrolls the way wizards can.
\item Bards \textit{can} use wizard scrolls, but this requires a Use Magic
Device check (with a slightly easier DC, per the modified skill description).
\end{itemize}
%

%
\subsection{Cleric}

\subsubsection{Spells}

Rather than having access to every cleric spell ever invented, clerics
(and other unlimited-library casters) know a large but finite number of
spells.

A starting cleric knows 6~+~(Wis~bonus) orisons and 4~+~(Wis~bonus)
spells of any given level when access to that spell level is acquired.
At each level-up, one additional spell \textit{of each spell level} is
gained.

At the DM's discretion, additional spells may be learned via story-specific
training during down-time.

Since a very large number of spells have been invented (any that would
have been a use for), you can make up a spell effect and ask the DM if it
would be available as a spell gained at character creation or during
level-up. Any such spells are subject to DM adjudication.

\subsubsection{Healing Domain}

Instead of Cure Wounds spells, the Healing domain's bonus spells for
spell levels 1-4 are:

\begin{tabular}{|l|l|}\hline
SL1 & Diagnose Disease \\
 & Diagnose Injury \\
 & Diagnose Poison \\
SL2 & Delay Poison \\
SL3 & Remove Disease \\
SL4 & Neutralize Poison \\ \hline
\end{tabular}

\subsubsection{Channel Energy}

A high Charisma score adds Channel Energy uses for a cleric. A low Charisma
score \textbf{does not} decrease it.
%

%
\subsection{Druid}

\subsubsection{Spells}

Rather than having access to every druid spell ever invented, druids
(and other unlimited-library casters) know a large but finite number of
spells.

A starting druid knows 6~+~(Wis~bonus) orisons and 4~+~(Wis~bonus)
spells of any given level when access to that spell level is acquired.
At each level-up, one additional spell \textit{of each spell level} is
gained.

At the DM's discretion, additional spells may be learned via story-specific
training during down-time.

Since a very large number of spells have been invented (any that would
have been a use for), you can make up a spell effect and ask the DM if it
would be available as a spell gained at character creation or during
level-up. Any such spells are subject to DM adjudication.

\subsubsection{Animal Companion}

A druid's animal companion bond uses a part of the druid's mind, as with
wizards' familiars. As a consequence, only one animal can be bonded at a
time, and it is mutually exclusive with certain other types of bond (as
described in the ``Wizard'' section).

Animal companions may not increase their Intelligence beyond 2 (``smart
animal'').

Animal companions understand their owners to the extent that a smart and
well-trained dog would (to the degree that their Int score allows). Their
owner understands the animal companion to the extent that a skilled animal
trainer or zookeeper would understand an animal they've worked with for
many years. Anything more than that requires magical assistance.

\subsubsection{Wild Shape}

A druid can freely assume the form of Tiny animals (such as cats) and
Diminutive animals (such as bats or squirrels), in addition to Small and
Medium animals.
%

%
\subsection{Paladin}

The various paladin organizations have strong beliefs about how the world
should work, and their members draw power from being exemplars of those
beliefs. For campaigns that use ``alignment'', they can be any Lawful
alignment:

\begin{itemize}
\item An example of a LG paladin would be Superman when he's in full ``boy
scout'' mode.
\item An example of a LN paladin would be Judge Dredd.
\item An example of a LE paladin would be Darth Vader (original trilogy
version).
\end{itemize}

Mechanical changes are as follows:

\begin{itemize}
%
\item The ``Detect Evil'' power instead acts as the ``Know Alignment''
spell (per the ``New Spells'' section).
%
\item The ``Smite Evil'' power works against any target that is strongly
opposed to the paladin's ethos, interpreted at the DM's discretion. This
adds the paladin's class level \textbf{and} their charisma bonus to
\textbf{both} the to-hit roll \textbf{and} the damage roll. Damage reduction
is only bypassed for Immaterium-powered entities such as undead, ghosts,
posession victims, and so forth (interpreted at the DM's discretion).
%
\item Paladins can spend a feat to take the ``Armour Training'' fighter
ability (prerequisite: Paladin level 3).
%
\item ``Lay On Hands'' has floor(level/2) + cha bonus uses, and does
floor(level/2) dice of healing. As with other healing magic, it heals
(roll result) vigor points \textit{and} (number of dice) wound points.
%
\item ``Divine Health'' doesn't make you immune to disease. It gives you
a bonus of +(5 + paladin level) to saving throws to resist disease and to
recover from disease, and a bonus of +(paladin level) to saving throws to
resist or recover from poison.
%
\item ``Divine Bond'''s weapon option doesn't involve a spirit; it's an
Immaterium effect, just as with Divine spellcasting. Per the SRD's text,
the weapon enhancement from Divine Bond stacks with magical weapon
enhancement, and if the weapon already has a +1 enhancement, a special
effect can be chosen rather than needing to add an enhancement bonus first.
Stacking only works with permanent enhancement bonuses, not temporary
bonuses (such as those granted by the ``Magic Weapon'' spell).
%
\item ``Divine Bond'''s animal option functions as a druid's animal
companion, per the SRD's text. For animal companion statistics, use
(paladin level - 4) as the equivalent druid level. Unlike an animal
companion, a paladin's mount \textit{does} have human-level intelligence;
consult the wizard's familiar table for the base intelligence score, using
(paladin level - 4) as the equivalent wizard level. The ``Empathic Link'',
`Speak With Master'', and ``Speak With Animals'' effects also gained (but
no other familiar effects). As with animal companions and familiars, only
one animal may be bonded at a time and the bond is mutually exclusive with
certain other types of bond. Divine bond animals cannot teleport.
%
\end{itemize}

\subsubsection{Spells}

Rather than having access to every paladin spell ever invented, paladins
(and other unlimited-library casters) know a large but finite number of
spells.

A starting paladin knows 6~+~(Cha~bonus) orisons and 4~+~(Cha~bonus)
spells of any given level when access to that spell level is acquired.
At each level-up, one additional spell \textit{of each spell level} is
gained.

At the DM's discretion, additional spells may be learned via story-specific
training during down-time.

Since a very large number of spells have been invented (any that would
have been a use for), you can make up a spell effect and ask the DM if it
would be available as a spell gained at character creation or during
level-up. Any such spells are subject to DM adjudication.
%

%
\subsection{Ranger}

\subsubsection{Spells}

Rather than having access to every ranger spell ever invented, rangers
(and other unlimited-library casters) know a large but finite number of
spells.

A starting ranger knows 4~+~(Wis~bonus) spells of any given level when
access to that spell level is acquired. At each level-up, one additional
spell \textit{of each spell level} is gained.

At the DM's discretion, additional spells may be learned via story-specific
training during down-time.

Since a very large number of spells have been invented (any that would
have been a use for), you can make up a spell effect and ask the DM if it
would be available as a spell gained at character creation or during
level-up. Any such spells are subject to DM adjudication.

\subsubsection{Animal Companion}

The ``animal companion'' version of ``Hunter's Bond'' functions as a
druid's animal companion, per the SRD's text. All of the limitations
involved with a druid's animal companion apply. For animal companion
statistics, use (ranger level - 3) as the equivalent druid level.
%

%
\subsection{Sorcerer}

Sorcerers are charisma casters, casting by \textit{intuition} rather than
using formal training, so there are limitations with respect to magic items:

\begin{itemize}
\item Sorcerers can't make scrolls the way wizards can.
\item Sorcerers \textit{can} use wizard scrolls, but this requires a Use
Magic Device check (with a slightly easier DC, per the modified skill
description).
\end{itemize}
%

%
\subsection{Wizard}

\subsubsection{Spells}

Rather than having access to every cantrip ever invented, wizards start
with a large but finite number of them.

A starting wizard may know up to 6~+~(Int~bonus) cantrips. At each level-up,
one additional cantrip may be learned \textit{in addition to} the normal
two spells per level-up.

Since a very large number of spells have been invented (any that would
have been a use for), you can make up a spell effect and ask the DM if it
would be available as a spell gained at character creation or during
level-up. Any such spells are subject to DM adjudication.

\subsubsection{Familiars and Bonded Objects}

A wizard's familiar gains intelligence by using its master's mind to think
with. A wizard may only support one bond of this type at a time; this is
why a wizard may only have one familiar, and why having a familiar excludes
other types of bond (such as animal companions) and some spells (such as
Unseen Servant, for campaigns that allow it).

A wizard's bonded object uses similar parts of its owner's mind to grant
its additional spell, so it is likewise mutually exclusive with familiar-type
bonds.

A familiar's ``Speak With Master'' and ``Speak With Animals'' effects are
present from level 1. Bear in mind that the starting attitude of other
animals is likely to be ``indifferent'' or worse.

\subsubsection{Spellbooks}

Adding spells to a spellbook is cheaper than in the Core Rulebook, but a
wizard has to pay for \textit{all} spells, \textit{including} spells gained
at character creation and during level-ups.

The cost of scribing a spell is $\mathrm{SL}^2 \times 10~\mathrm{sp}$.
Scribing a cantrip costs 5~sp.

A spell takes up a number of spellbook pages equal to $\mathrm{SL}^2$.
A cantrip takes up a single page. (Very high level characters will have
trouble carting their entire libraries around, but this is not likely to
be a problem in a realistic campaign.)
%

%
%
\section{New Classes}

%
\subsection{Expert (Magical)}

This is a variant of the NPC ``Expert'' class that has spellcasting.

Progression and features are per the Expert class, with the following
changes:

\begin{itemize}
\item The EM gets 8 class skills rather than 10.
\item The EM gets 4+Int skill ranks per level, rather than 6 (they
additionally get 1 bonus rank replacing ``favoured class'', as described
in ``character generation'').
\item The EM gains spells as if they were a Cleric, Druid, or Wizard one
level lower than their EM level. They do not gain any of the other class
features (at DM's discretion some of these might be bought as feats).
\item At level 1, the EM has 3 orison or cantrip spell slots.
\end{itemize}
%

Typical professions would be porters/teamsters, chefs, alchemists, and
doctors/paramedics.

In the 2022 campaign, many industries were staffed by wizards (alchemy) or
clerics (food service), and there was a huge jump in capability between
``expert with a cantrip feat'' and ``full wizard/cleric''. This class
attempts to bridge that gap. Full clerics/druids/wizards will tend to be
specialists, with most of any given industry being magical experts instead.

%
% This is the end of the file.
