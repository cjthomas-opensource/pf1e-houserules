% "World of Adventure" campaign guide - Spells
% Written by Christopher Thomas.

\chapter{Spells}
\label{sect-spells}

%
%
\section{Modified Spells}

%
\subsection{Animate Rope}

Animate Rope lets you make a ranged touch attack with the rope as part of
the casting action, rather than requiring a separate action.

If it required a separate action, the spell would never get used.
%

%
\subsection{Ant Haul}

Per the CRB, this triple's the target's carrying capacity for land travel,
for 2~hours/level.

There is a version that's SL2 that triples the carrying capacity of a
flying creature, for 1~hour/level.
%

%
\subsection{Beast Shape}

You can assume Tiny forms with Beast Shape~I, and Diminutive forms with
Beast Shape~II.
%

%
\subsection{Calm Animals}

Calm Animals works on anything with the ``animal'' or ``magical beast''
type that does not have full sapience (i.e. Int less than 3). Be advised
that most things with Int~0 are immune to mind-affecting spells due to
being mindless (they're mostly vermin or oozes or the like, not animals).
The DM can adjudicate special cases (things that are animal-like but don't
have ``animal'' or ``magical beast'' type might still be affected).
%

%
\subsection{Calm Emotions}

This spell is also available as a Wizard~2 spell.

There is a version that's SL3 that lasts 1 minute per level and does not
require concentration.

There is a SL~4 version lasting 1 round per level and a SL~5 version
lasting 1 minute per level that give targets a -5 penalty to the saves to
resist the spell.
%

%
\subsection{Charm Animal}

Charm Animal works on anything with the ``animal'' or ``magical beast''
type that does not have full sapience (i.e. Int less than 3). Be advised
that most things with Int~0 are immune to mind-affecting spells due to
being mindless (they're mostly vermin or oozes or the like, not animals).
The DM can adjudicate special cases (things that are animal-like but don't
have ``animal'' or ``magical beast'' type might still be affected).

Rather than making the caster a ``trusted friend'', this spell improves
the target's attitude towards the caster by 2 steps. That's still probably
good enough, especially if you try using Handle Animal to improve it
further immediately afterward.

There is a version that's SL3 that gives the recipient a -5 penalty to
their Will save to resist the spell.
%

%
\subsection{Charm Person}

Charm Person works on anything sapient (from a human to a unicorn to a
dragon). The target \textit{does not} need to understand your language
to be affected. Bear in mind that powerful creatures typically have very
good Will saves.

Rather than making the caster a ``trusted friend'', this spell improves
the target's attitude towards the caster by 2 steps. That's still probably
good enough, especially if you try using Diplomacy to improve it further
immediately afterward.

There is a version that's SL3 that gives the recipient a -5 penalty to
their Will save to resist the spell.
%

%
\subsection{Clairvoyance}

See notes at ``Scrying''.
%

%
\subsection{Command}

Command works on anything that can understand you. This usually means
human-level intelligence, but Speak with Animals can circumvent that that.

The command is limited to one simple action, not one word. Typical
examples might be ``drop your weapons'', ``get out'', ``go away'',
``lie down'', and so forth.
%

%
\subsection{Comprehend Languages}

Comprehend Languages with written material works by accessing the
Immaterium.

This means it will work on living languages, but won't be able to translate
dead languages (no fluent speakers within the last 50-100 years).

It also means that it can only translate codes and ciphers that are widely
known and that don't require keys or other message-specific information.

There's a SL3 version of the spell that can translate writing that only a
\textit{few} people know how to read, but the information still has to be
around somewhere.
%

%
\subsection{Daylight}

Daylight normally lasts for 10 minutes per level, but if cast on an object
made of gold (or plated in gold), it lasts for 1 hour per level.

This is sometimes used for household and industrial lighting, in the same
manner as the Light cantrip, but the Light cantrip is much more common
(since one person can cast it on a large number of lamps).
%

%
\subsection{Darkness and Deeper Darkness}

These conjure jet-black spheres of complete darkness, rather than dimming
ambient light.
%

%
\subsection{Detect Evil}

Superseded by ``Know Alignment''.
%

%
\subsection{Discern Lies}

This is Cleric 3, Inquisitor 3, Paladin 2.

Since it doesn't last long, the normal use-case is to get written
statements or verbal testimony and then to ask ``is your
statement/testimony true and complete?''.
%

%
\subsection{Dispel Evil}

This targets creatures (and spell effects cast by creatures) that are
strongly opposed by your personal ethos, interpreted at the DM's discretion.
This is the same class of targets that would be affected by a paladin's
``Smite Evil'' effect.

Since there are no other planes, the spell instead dismisses conjured or
summoned entities that are valid targets (as with Banishment or Dismissal).
%

%
\subsection{Dominate Animal}

Dominate Animal works on anything with the ``animal'' or ``magical beast''
type that does not have full sapience (i.e. Int less than 3). Be advised
that most things with Int~0 are immune to mind-affecting spells due to
being mindless (they're mostly vermin or oozes or the like, not animals).
The DM can adjudicate special cases (things that are animal-like but don't
have ``animal'' or ``magical beast'' type might still be affected).

This spell lasts 1 minute per caster level, and the target gets a new
saving throw each minute.

There is a Wizard~3 version of this spell.

There is a version that's SL5 that lasts 1 day per caster level, with a new
saving throw each day. It also gives the recipient a -5 penalty on the Will
save to resist the spell.
%

%
\subsection{Dominate Person}

Dominate Person works on anything sapient (from a human to a unicorn to a
dragon). Bear in mind that powerful creatures typically have very good Will
saves.

This spell lasts 1 day per caster level, and the target gets a new saving
throw each day.

There is a version that's SL7 that gives the recipient a -5 penalty on the
Will save to resist the spell.
%

%
\subsection{Endure Elements}

The ``Mass'' version of this spell is SL2, rather than the usual SL3.

%
\subsection{Expeditious Retreat}

Expeditious Retreat gives a land speed bonus equal to your base speed.
Per the CRB it's 1~minute/level and bard/wizard~1.

There is a version that's SL3 that gives a speed bonus to all movement
modes (equal to each mode's base speed).

This is slightly less useful than Haste (which is also SL3), but lasts
longer.
%

%
\subsection{Floating Disk}

Floating Disk has several standard shapes; the caster chooses a shape at
the time of casting, depending on what type of cargo is being moved.

For moving boxes, crates, and miscellaneous goods, the standard shapes are
a 3' diameter disk, a 2.5' square, or a 2' by 3.5' rectangle. There are
variants with and without a 4" tall lip at the edge.

For moving liquids, standard shapes are a hemispher 2' in diameter (volume
4 cubic feet), and a cylinder 2' wide and 1' high (volume 3 cubic feet).
Water weighs 55~lbs per cubic foot. One cubic foot is about 7.5 gallons.

The weight limit is 100~lbs per caster level. If the weight limit is
exceeded, the spell fails and the disk winks out. The disc can be raised and
lowered, from touching the ground to 3 feet above ground level.

This spell is widely used in warehouses for moving freight, for loading and
unloading carts, and even in place of carts when moving small amounts of
cargo (anything you'd use a pallet truck for in our world). Standard pallets
are built so that a rectangular disk can fit under the pallet without the
pallet sliding off.

There is a version that's SL2 that has 10 times the weight limit, can be
raised to 10 feet above ground level, and that lasts 2 hours per caster
level. This is used for moving heavy cargo crates in the field, and for
loading and unloading cargo from high shelves (anything you'd use a forklift
for in our world).

There is a ``Mass Floating Disk'' spell that's SL3 and gives one disk per
caster level (all controlled by the caster), and a ``Mass Freight Pallet''
spell that's SL4 and gives one heavy-duty disk per caster level.
%

%
\subsection{Geas}

Geas works on anything sapient (from a human to a unicorn to a dragon).
The target must be able to understand you to be affected. Bear in mind that
powerful creatures typically have very good Will saves.

These spells give an instruction to do or refrain from doing some general
task or activity. There is an implicit ``to the extent that you are able''
added to instructions (pausing to eat and sleep are okay), and there is an
implicit ``to the extent that you can without it being suicide'' as well.
One or both of these is often made explicit (e.g. ``from sunrise to sunset,
six days out of seven, you will labour towards X'').

On breaking the direction, the subject is immediately Sickened and gets a
penalty to all ability scores (without delay). The penalty gets worse (by
the spell's penalty increment) every 24 hours, up to the spell's penalty
limit. This cannot reduce an ability score below 3. When the spell's
duration expires, or when the subject sincerely attempts to abide by its
terms again, symptoms reduce in severity every 24 hours (with the Sickened
condition disappearing when the last of the ability penalties do).

Instructions that are prohibitions (such as ``don't kill anyone'') usually
include terms of restitution, so killing someone in anger and then resolving
not to do it again isn't enough to remove the penalty -- the recipient has
to sincerely intend to make restitution (or to accept a pre-specified
alternate punishment) as well.

Lesser Geas is Bard~3, Cleric~4, Wizard~4.
Greater Geas is Bard~5, Cleric~6, Wizard~6.

Both spells have a casting time of 1 minute (long enough to spell out the
task or prohibition). Both spells allow a Will save; for Greater Geas, the
save has a -5 penalty. There is no hit dice limit; if an ancient dragon
fails its saving throw (and fails to kill you while you're casting the
spell), it's affected.

Lesser Geas lasts 2 days per caster level, and applies an ability score
penalty of -2 to 24 hours, to a limit of -8.

Greater Geas lasts 2 weeks per caster level, and applies an ability score
penalty of -3 per 24 hours, to a limit of -12.
%

%
\subsection{Hide from Animals}

Hide from Animals works on anything with animal intelligence or less that
could be described as a ``critter''. This includes insects, magical beasts,
and oozes, but not including undead, outsiders, or aberrations.
%

%
\subsection{Light}

Light normally lasts for 10 minutes per level, but if cast on an object
made of gold (or plated in gold), it lasts for 1 hour per level.

This is widely used for household lighting, and handheld objects with
plated fiddly-bits and reflectors are readily available in ``flashlight''
and ``lantern'' form-factors.
%

%
\subsection{Longstrider}

Longstrider gives a land speed bonus of half your base speed, rounded down
to the next multiple of 5'. Per the CRB it's 1~hour/level and
druid/ranger~1.

There is a version that's SL2 that gives a speed bonus to all movement
modes (equal to half each mode's base speed, rounded down per above).
%

%
\subsection{Mage Armor}

This gives you a +4 armour bonus, stacking with the bonus from Shield. As
noted, this is effective even against incorporeal entities.
%

%
\subsection{Mage's Private Sanctum}

The level, target, and effect of this spell are adjusted:

\begin{itemize}
\item It's Cleric 4, Wizard 4.
\item The area of effect is either a sphere centered on a location touched
with a radius chosen by the caster (up to 40'), or the interior of a
building with up to 30,000 square feet of floor area (1200 5' squares).
\item Instead of the effects described in the Core Rulebook and SRD, the
affected area behaves as if under the effects of the ``Privacy Screen''
spell and the ``Nondetection'' spell.
\end{itemize}

For protection against mind-reading, use the ``Protection from Mental
Control'' spell.
%

%
\subsection{Magic Circle against Evil}

As with ``Protection from Evil'', this has been split into two spells.
See ``Protection from Immaterial Influence'' and ``Protection from Metal
Control'' in the ``New Spells'' section.
%

%
\subsection{Magic Fang}

Magic Fang works as-described in the CRB (giving +1 to a single natural
weapon).

There is a version that's SL2 that affects all natural weapons.

Greater Magic Fang works on all natural weapons and gives a scaling bonus
(+1 per two full levels, to a maximum of +5).
%

%
\subsection{Magic Missile}

If a caster gets multiple missiles, and several missiles are directed at
one target, those missiles' damage is added together before DR is applied
(in the same manner as with the ``Clustered Shots'' feat).
%

%
\subsection{Mount}

In addition to being ridden, horses conjured by this spell can be used as
light draft animals.

For heavier work, the ``Conjure Team'' spell is used.
%

%
\subsection{Nondetection}

There is a ``Greater'' version of Nondetection that increases the DC by
5, and is also more effective at degrading scrying. See notes at
``Scrying''. Greater Nondetection is Wizard~5.
%

%
\subsection{Phantom Steed}

The special effects kick in at levels 7, 8, 9, and 10 (rather than 8, 10,
12, and 14).
%

%
\subsection{Protection from Evil}

This spell has been split into ``Protection from Immaterial Influence''
and ``Protection from Metal Control'' (see the ``New Spells'' section).
%

%
\subsection{Purify Food and Drink}

This does not affect poisons.

This removes spoilage that is present in food, resulting in less food.
There is a new spell (``Preserve Food and Drink'') that prevents spoilage.
%

%
\subsection{Scrying}

Sending and Scrying can operate at long range, but are limited by
interference from high-background areas (with a useful range of 100-200
miles). The spells can also be attenuated by lead.

Signal levels are Decent, Degraded, Very Bad, and None.

There are higher-level (``Greater'') versions of Sending, Scrying, and
Clairvoyance that operate more effectively. These are two levels higher
than the corresponding standard spells.

Certain spells (or enchanted devices with those spells) at the target can
be used to enhance the signal. These are used as communications aids (for
Sending) and for espionage (Scrying and Clairvoyance). See ``Sending Focus''
and ``Scrying Focus'' for details.

Summary of spells and countermeasures affecting signal:

\begin{tabular}{l|l}\hline
\textbf{Effect} & \textbf{Conditions} \\ \hline
%
baseline & Sending (C4/W5), Scrying (D4/C5/W5), Clairvoyance (W3) \\
\hline
%
+1 signal & Greater version of the spell. \\
+1 signal & Sending or scrying focus at the target. \\
+2 signal & Greater focus at the target. \\
\hline
%
-1 signal & Lead foil. \\
-2 signal & Heavy lead plates. \\
-2 signal & Nondetection (W3) (Scrying/Clairvoyance only). \\
-3 signal & Heavy multi-layer shielding. \\
-4 signal & Greater Nondetection (W5) (Scrying/Clairvoyance only). \\
\hline
%
\end{tabular}
%

%
\subsection{Secure Shelter}

This spell does not provide an ``Unseen Servant'' (see below).
%

%
\subsection{See Alignment}

Superseded by ``Know Alignment''.
%

%
\subsection{See Invisibility}

In addition to seeing invisible things, the user can also see through
illusions of the ``figment'' or ``glamer'' type, such as those produced by
Disguise Self or Minor Image. The user perceives both the illusion and
whatever's under the illusion, and knows which is which.

Illusions of the ``pattern'' type (such as Colour Spray) are identified as
illusions but still do their thing (mind-affecting).

Illusions of the ``phantasm'' type (all in your head) function normally.

Illusions of the ``shadow'' type (conjuring in shadow-matter) are
identified as what they are (typically resulting in reduced damage).
%

%
\subsection{Sending}

This has a casting time of 1 minute, rather than 10 minutes.

The recipient of a ``Sending'' call can choose to reject the call. The
person making the call can't tell if the all was rejected or if the
contact attempt failed for some other reason.

See also the notes at ``Scrying''.
%

%
\subsection{Shield}

This gives you a +4 armour bonus, stacking with the bonus from Mage Armor.
As noted, this is effective even against incorporeal entities. This spell
\textbf{does not} negate Magic Missiles, though the damage reduction will
certainly help.
%

%
\subsection{Spiritual Weapon}

Spiritual Weapon's damage bonus is +1 per caster level (rather than per
3 levels).

Otherwise it has a very hard time dealing with opponents that have DR.
%

%
\subsection{Suggestion}

Suggestion works on anything that can understand you. This usually means
human-level intelligence, but Speak with Animals can circumvent that that.
Bear in mind that powerful creatures typically have very good Will saves.
%

%
\subsection{Tiny Hut}

This has a 10' radius (20' diameter). That still has almost as much square
footage as my apartment.
%

%
\subsection{Zone of Truth}

This is a Divination spell rather than an Enchantment spell (per Discern
Lies).

Rather than preventing lying, this spell produces a visible indicator of
spoken lies (traditionally a puff of mist expelled as the lie is spoken,
though other variants of the spell could be made).

Since a Will save prevents the spell from detecting lies, the normal
use-case involves asking everyone giving testimony to speak a lie (to a
question like ``what day is today''), to confirm that they are affected.

Enchanted devices with this effect are widely used in court.
%

%
\subsection{Dimensional and Planar Magic}

Spells that involve creating pocket dimensions or that involve travel to
other planes do not exist in this setting. Some of these are removed (per
the next section); others remain but are re-flavoured as operating by
other means. A list of re-flavoured Core Rulebook spells in this category
is as follows:

\begin{itemize}
\item \textbf{Banishment} --
This dismisses conjured/summoned entities.
\item \textbf{Contact Other Plane} --
This tries to find the requested information in the gestalt knowledge of
the Immaterium.
\item \textbf{Dismissal} --
This dismisses conjured/summoned entities.
\item \textbf{Ethereal Jaunt} --
This makes the targets intangible, not ethereal.
\item \textbf{Etherealness} --
This makes the targets intangible, not ethereal.
\item \textbf{Imprisonment} --
The target is moved to its entombment location the old-fashioned way, not
by teleporting there. Sufficiently determined miners could follow the
disturbed earth/stone.
\item \textbf{Planar Ally} --
This makes a magically-conjured body to house an Immaterial concept, as
with other summoning spells.
\item \textbf{Planar Binding} --
This makes a magically-conjured body to house an Immaterial concept, as
with other summoning spells.
\end{itemize}
%

%
%
\section{Removed Spells}

%
\subsection{Charm Monster}

This spell does not exist. Use Charm Animal for non-sapient targets and
Charm Person for sapient targets.
%

%
\subsection{Continual Flame}

This spell does not exist; enchanted devices serve the same function.
%

%
\subsection{Read Magic}

Anyone taught Arcane or Divine magic is also taught to read Arcane and
Divine spell notation. A Spellcraft roll may be needed to decipher an
unfamiliar spell (this automatically succeeds if a person familiar with the
spell is available to provide instruction).

Spell notation forms used by unfamiliar cultures typically require both
Comprehend Languages and a more difficult Spellcraft roll if an interpreter
is not available.
%

%
\subsection{Spell Resistance}

I am not using Spell Resistance, so this spell is also removed. If you
choose to use SR in your campaign, this spell would exist as normal.
%

%
\subsection{Unseen Servant}

This spell is rarely used \textit{and} has too many headaches to adjudicate
to be worth it.

For campaigns that use it, it would work the same way familiars do: by using
the caster's mind to think with. As such it would not be compatible with
having a familiar, animal companion, or bonded object.
%

%
\subsection{Dimensional and Planar Magic}

Spells that involve creating pocket dimensions or that involve travel to
other planes do not exist in this setting. A list of Core Rulebook spells
in this category are:

\begin{itemize}
\item Astral Projection
\item Blink
\item Dimensional Anchor
\item Dimensional Lock
\item Forbiddance
\item Mage's Magnificent Mansion
\item Phase Door
\item Plane Shift
\item Rope Trick
\item Secret Chest
\end{itemize}
%

%
\subsection{Teleportation Magic}

Spells that involve teleportation do not exist in this setting. A list of
Core Rulebook spells in this category are:

\begin{itemize}
\item Dimension Door
\item Instant Summons
\item Teleport (all variants)
\item Word of Recall
\end{itemize}
%

%
%
\section{New Spells}

% FIXME: Plant diagnostics, healing, yield increase, grow in unfertile.

%
\subsection{Apprentice's Spinner}

Wizard 0

V/S, 10 minutes per caster level

This makes an object spin, with force comparable to what you'd get with a
child's spinning top, or (if geared down) turning a screwdriver.

This has little use industrially, but sees niche household uses (light-duty
sewing machines, record players, and the equivalent of wind-up toys). Clocks
are sometimes enchanted with this effect to be self-winding.

\textit{NOTE: In some campaigns, casting this on an appropriately built
mechanism or a flywheel made from an appropriate material can extend the
duration to 1 hour per caster level, much as gold extends the Light cantrip.
This is an optional change.}
%

%
\subsection{Bardic Masterpiece: Scare}

Flavoured as ``Terrifying Tale'' or ``Haunting Melody'' or similar.

Prerequisite: Perform (6 ranks)

Cost: Buy it with a feat, or as a L3 Bard spell.

Effect:

At the end of the round in which you initiate the performance, a set of
targets that you specify must make a Will saving throw. On failure, the
creatures become Frightened; on success, they become Shaken. You can affect
a number of targets equal to your Charisma modifier. Targets must be able
to hear and understand your performance.

When the performance ends, affected creatures may make new saving throws.
Saving throws are made at the start of a creature's turn. A successful
save ends the condition. Attempts may be made each round until successful.

Use:

2 Bardic Performance rounds to initiate, plus 2 Bardic Performance rounds
per target per round that the effect is maintained. This includes the
round in which the performance is initiated.

For purposes of computing the save DC, this is a L2 effect (not L3).
%

%
\subsection{Bind Wound}

Cleric 0, Druid 0

V/S/DF, one wound, 24 hours, Fortitude negates (harmless)

This cleans one open wound and applies a magically conjured covering to
protect the wound and hold it closed. Rough treatment may open the wound
again or remove the covering.

This does more or less the same thing as real-world medical glue or
spray-on bandages.

Non-magical cleaning and bandaging can accomplish the same thing, but the
spell does not require a skill check or medical supplies.
%

%
\subsection{Binding Contract}

Cleric 3, Wizard 3

V/S/M, one living creature, indefinite, n/a save (see below)
Material component is 5000~sp worth of diamonds.

Casting time is anywhere from a few rounds to several minutes; it involves
reading out the contract and making sure it's understood.

This spell enforces the terms of a contract on the recipient. The recipient
must understand and willingly agree to abide by the terms of the contract
for the spell to take effect; no saving throw is made (they voluntarily
failed it). If the recipient does not agree to be bound by the contract,
the spell is expended but the material components are not.

The recipient has a general idea if a given action or inaction would break
the contract; this makes it hard to accidentally break it.

The contracting parties both know the current compliance state and are
updated about changes to compliance.

Immediately upon breaching the terms of the contract, the recipient is
Sickened, and gets a -2 penalty to all ability scores (without delay).
The penalty gets worse (by -2) every 24 hours, to a maximum of -8. This
cannot reduce an ability score below 3. The penalty remains until
restitution for all outstanding breaches is made. When no outstanding
breaches remain, symptoms reduce in severity every 24 hours (with the
Sickened condition disappearing when the last of the ability penalties do).

Terms of restitution for breaches are usually spelled out in the contract.
The person who offered the contract can forgive the recipient of one or
more breaches at their discretion.

The contract remains in force indefinitely, irrespective of whether
penalties have occurred and of whether breaches are outstanding or forgiven.
The contract ends when the recipient dies (which the offerer immediately
knows about), or when the person who offered the contract explicitly ends
the contract or dies (both of which the recipient immediately knows about).

There is a version that's SL~5 that affects a number of recipients up to
the caster level (with one set of shared terms). The material component for
this version of the spell is \mbox{20,000~sp} worth of diamonds (beakeven
point vs the SL~3 spell is 4 recipients).

The Harbourton version of this spell traditionally involves reading out the
offered terms, asking ``do you understand the offer?'', and then asking
``do you agree to this contract?''. Replying ``yes'' to both completes the
spell.
%

%
\subsection{Conjure Team}

Wizard 2

V/S, close (25' + 5'/2 levels), 6 or 12 hours (D) (see below)

This conjures either one or two draft animals (draft horses or oxen), in
the same manner as the ``Mount'' spell. If one animal is conjured, it lasts
for 12 hours; if two are conjured, they last for 6 hours.

These are typically used to draw ploughs, wagons, or large carriages.

As conjured creatures, they do not need to eat or sleep, but also aren't
capable of strenuous exertion (such as sprinting/galloping).
%

%
\subsection{Detect Background}

Cleric 0, Druid 0, Wizard 0

This tells the caster what the level of background magic around them is
(none, low, medium, high, or extreme).
%

%
\subsection{Delay Specific Poison}

bard 1, cleric 1, druid 1, paladin 1
V/S/DF, creature touched, 1 hour/level, Fortitude negates (harmless)

This is a collection of many different spells, not one spell. Each spell
functions as ``Delay Poison'', but only affecting one narrow class of
poisons (such as snake venom or specific classes of alchemical reagents).

These spells are widely used for first aid in situations where poisonous
substances or venemous creatures are likely to be encountered. Potions are
widely available for commonly-encountered types.

The ``Communal'' version of this spell is SL1 for rangers, SL2 for
everyone else (since rangers get non-specific ``Delay Poison'' as SL1).
%

%
\subsection{Diagnose Disease}

cleric 1, druid 1, ranger 1

V/S/DF, one subject per caster level, Fortitude negates (harmless)

This identifies mundane diseases that the subject is suffering from. This
does not identify poison or wounds, though infections within wounds will
be flagged by the spell.

This spell is used during triage and as a first step before magical or
non-magical treatment.

A non-magical medical examination can accomplish the same thing, but the
spell works immediately and does not require a skill check.
%

%
\subsection{Diagnose Injury}

cleric 1, druid 1, ranger 1

V/S/DF, one subject per caster level, Fortitude negates (harmless)

This identifies physical injuries that the subject is suffering from. This
does not identify poison or disease, though injuries made worse by these
will be flagged by the spell.

This spell is used during triage and as a first step before magical or
non-magical treatment.

A non-magical medical examination can accomplish the same thing, but the
spell works immediately and does not require a skill check.
%

%
\subsection{Diagnose Poison}

cleric 1, druid 1, ranger 1

V/S/DF, one subject per caster level, Fortitude negates (harmless)

This identifies mundane poisons that are present within the subject, and
damage caused by those poisons. In mechanical terms, the caster knows
how many more saving throw successes are needed to end the poisoning (if
the poisoning is ongoing), whether poison is present but delayed by Delay
Poison, or whether poison is present but has already caused as much damage
as it's going to (saves already successful).

This spell is used during triage and as a first step before magical or
non-magical treatment.

A non-magical medical examination can accomplish the same thing, but the
spell works immediately and does not require a skill check.
%

%
\subsection{Know Alignment}

Bard 1, Cleric 1, Wizard 1

60' cone, concentration 1 minute/level, Will negates

This spell produces an emanation similar to that of ``Detect Thoughts''.
The caster can focus on any person that they can see or otherwise perceive
within the target area, and learn about their world-view on the good/evil
and law/chaos axes.

Information starts off very vague and gets more precise with continued
inspection (up to 3 rounds).
%

%
\subsection{Magic Circle against Immaterial Influence}

Cleric 3, Inquisitor 3, Paladin 3, Wizard 3
person or object touched, 10 minutes/level, V/S/DF

At the time of casting, all creatures within a 10' emanation of the the
person or object touched are granted the effects of ``Protection from
Immaterial Influence''. The effect lasts as long as they remain within the
emanation (until the spell's duration expires).

An alternate version of the spell exists (a separate spell) that is used
for containment. It is cast on a circle, and while the spell is active
grants PfImIn's protections to people outside the circle when resisting
effects or attacks that come from inside the circle.

There is a version that's SL5 that provides a -4 attack roll penalty and
a +4 saving throw bonus.

``Communal'' and ``Mass'' versions of this spell could be made, and might
or might not already be invented/available at the DM's discretion.
%

%
\subsection{Magic Circle against Mental Control}

Cleric 3, Inquisitor 3, Paladin 3, Wizard 3
person or object touched, 10 minutes/level, V/S/DF

At the time of casting, all creatures within a 10' emanation of the the
person or object touched are granted the effects of ``Protection from Mental
Control''. The effect lasts as long as they remain within the emanation
(until the spell's duration expires).

An alternate version of the spell exists (a separate spell) that is used
for containment. It is cast on a circle, and while the spell is active
grants PfMC's saving throw bonuses to people outside the circle when
resisting effects that come from inside the circle.

There is a version that's SL5 that provides a +10 saving throw bonus.

``Communal'' and ``Mass'' versions of this spell could be made, and might
or might not already be invented/available at the DM's discretion.
%

%
\subsection{Neutralize Specific Poison}

bard 3, cleric 3, druid 2, ranger 2, paladin 3
V/S/DF, creature touched, Fortitude negates (harmless)

This is a collection of many different spells, not one spell. Each spell
functions as ``Neutralize Poison'', but only affecting one narrow class of
poisons (such as snake venom or specific classes of alchemical reagents).
%

%
\subsection{Padded Weapon}

Cleric 1, Wizard 1

touch, 1 weapon or 10 projectiles, 1 hour per caster level

This makes a normally-lethal weapon function deal nonlethal damage.

The damage type becomes ``bludgeoning'', and the damage die size is
reduced by one step (1d8 becomes 1d6, etc).

Most versions of this spell cause a visible aura around the weapon, so
that people on the receiving end know that they are being attacked with
nonlethal force. This spell is widely used for police work.

It is normally more cost-effective to buy a purpose-built nonlethal weapon,
but when flexibility is desired, this spell provides a nonlethal option.
%

%
\subsection{Preserve Food and Drink}

Cleric 0, Druid 0
10', 1 cubic foot per level of food and water, 24 hours

This spell prevents the targetted food and drink from spoiling. It may still
suffer contamination, so packaging it properly is advised if it's being
transported.
%

%
\subsection{Privacy Screen}

Cleric 2, Wizard 2
location touched, see text, 10 minutes/level (D)

This spell affects either a sphere centered on a location touched with a
radius chosen by the caster (up to 20'), or the interior of a room with up
to 1200 square feet (48 5' squares).

The spell prevents people inside and outside the area from seeing or hearing
each other. There are several ways it might do this; a way is chosen at the
time of casting:
\begin{itemize}
\item The boundary of the area may be opaque (typically looking like
shimmering black velvet for the standard version; the Jewelled Empire's
version looks like the walls of a silk tent).
\item The boundary of the area may instead distort light in the manner of
textured glass (showing the rough size and shape of people on the other
side but obscuring all other details).
\item The boundary completely deadens sound.
\item The boundary may optionally produce sound near itself (rushing water
for the standard version).
\end{itemize}

The appearance and sound of the boundary can be identified as illusions by
appropriate magic (such as See Invisibility), but are not penetrated by
such magic.
%

%
\subsection{Protection from Background Magic}

Cleric 1, Druid 1, Ranger 1, Wizard 1

touch, 24 hours, Will negates (harmless)

This protects a person from background magic, reducing its effective
strength by one step.

The ``Mass'' version is SL2, rather than the usual SL3.

There is a version that's SL2 that reduces the effective background
strength by two steps. The ``Mass'' version of that spell is SL3.
%

%
\subsection{Protection from Immaterial Influence}

Cleric 1, Inquisitor 1, Paladin 1, Wizard 1
touch, 1 minute/level, V/S/DF

This spell reduces the extent to which the Immaterial realm can influence
the warded creature. This has several effects:

\begin{itemize}
%
\item Summoned creatures, ghosts, and other entities that are ``concepts
from the Immaterial Realm cothed in magically conjured bodies'' cannot
physically touch a creature protected by the spell. Corporeal undead can
still do so, as they are mundane matter animated by magic rather than
conjured matter.
%
\item Entities such as those described above, along with corporeal undead
and creatures posessed by immaterial entities, receive a -2 to attack rolls
against the protected target. The protected target gets a +2 resistance
bonus to saving throws made against effects that these entities generate.
%
\item The protected target cannot be drawn into shared dreams or visions.
%
\item The severity of immaterial influence on the target due to being a
public figure (with lots of people thinking about them) is reduced by one
step.
%
\end{itemize}

There is a version that's SL3 that provides a -4 attack roll penalty, a
+4 saving throw bonus, and that reduces the severity of immaterial
influence by 3 steps.

``Communal'' and ``Mass'' versions of this spell exist.
%

%
\subsection{Protection from Mental Control}

Cleric 1, Inquisitor 1, Paladin 1, Wizard 1
touch, 1 minute/level, V/S/DF

This provides the target with a +5 saving throw bonus (resistance bonus)
against Charm, Command, Suggestion, Dominate, and similar spells. If the
spells are already in effect, they get a new saving throw.

The target also receives a +5 bonus to saving throws (resistance bonus)
against Detect Thoughts and similar mind-reading spells.

There is a version that's SL3 that provides a +10 saving throw bonus.

``Communal'' and ``Mass'' versions of this spell exist.
%

%
\subsection{Remove Specific Disease}

cleric 2, druid 2, ranger 2
V/S/DF, creature touched, Fortitude negates (harmless)

This is a collection of many different spells, not one spell. Each spell
functions as ``Remove Disease'', but only affecting one narrow class of
diseases.
%

%
\subsection{Scrying Focus}

Druid 2, Cleric 2, Wizard 2

touch, 24 hours, Will negates

This makes a person, place, or object easier to target with the ``Scrying''
and ``Clairvoyance'' spells, per notes at ``Scrying''.

There is a ``Greater'' version at SL4 that is even more effective.
%

%
\subsection{Sending Focus}

Cleric 2, Wizard 2

touch, 24 hours, Will negates

This makes a person, place, or object easier to target with the ``Sending''
spell, per notes at ``Sending''. Signal quality is improved, mitigating
interference.

There is a ``Greater'' version at SL4 that is even more effective.
%

%
\subsection{Smaw's Pencil}

Cleric 2, Wizard 2
V/S/M, filler rods touched, 10 minutes per caster level

This performs stick-welding. Metal filler rods are consumed during welding.
Dipping them in alchemical flux paste is wise if you want high-strength
welds.

You can gouge (cut metal) by using charcoal sticks instead of filler rods.
These are consumed very quickly (ten times as much charcoal consumed as
metal removed, by volume).

You can braze by using a lower-melting filler rod. The target site heats up
to near the melting point of the filler rod, not the melting point of the
workpiece.

You'll need to wear welding goggles to avoid temporary (or permanent)
blindness. Flame-rated goggles will do, since you're only dealing with hot
metal, not an arc.

Enchanted devices with this effect exist, but are very expensive.
%

%
\subsection{Smaw's Visor}

Cleric 1, Wizard 1
V/S, 10 minutes per caster level

This protect's the caster's eyes from bright light, the same way an
automatically-darkening welding mask does (putting a filter over the
caster's eyes). Much as with a welding mask, all light is attenuated, not
just the bright spots.

Whether darkened or not, the spell also filters near-infrared, giving
everything a slight greenish cast and preventing eye damage from looking
at hot metal or forge-coals.

The intended use-case is stick-welding (using ``Smaw's Pencil''), but it's
also helpful for forge work and may have niche uses elsewhere.

Enchanted devices with this effect exist but are uncommon, as it's usually
cheaper to use mundane welding goggles or visors if the spell isn't
available.
%

%
\subsection{Unseen Oxen}

Wizard 2

V/S, 10 minutes per caster level

This makes a heavy object such as a millstone or large flywheel spin, with
force comparable to what you'd get with a pair of oxen powering a mill or
pulling a wagon.

This can be used to power heavy industrial equipment or as the ``prime
mover'' of a small factory. It can also be used to make self-propelled
wagons, though conjured draft animals are usually more practical.

\textit{NOTE: In some campaigns, casting this on an appropriately built
mechanism or a flywheel made from an appropriate material can extend the
duration to 1 hour per caster level, much as gold extends the Light cantrip.
This is an optional change.}
%

%
\subsection{Wizard's Wheel}

Wizard 1

V/S, 10 minutes per caster level

This makes an object up to the size of a cart-wheel spin, with force
comparable to what you'd get pedalling a bicycle or pumping a treadle.

This can be used to power light industrial equipment or heavy household
machinery. It can also be used to make self-propelled carts or scooters,
though conjuring a mount is usually more practical.

\textit{NOTE: In some campaigns, casting this on an appropriately built
mechanism or a flywheel made from an appropriate material can extend the
duration to 1 hour per caster level, much as gold extends the Light cantrip.
This is an optional change.}
%

%
% This is the end of the file.
