% "World of Adventure" campaign guide - Equipment
% Written by Christopher Thomas.

\chapter{Equipment}
\label{sect-equip}

%
%
\section{Masterworking}

\fixme{Consolidate rules.}

\fixme{Add sqrt(cost * 100) rule.}

%
%
\section{Armour}

\fixme{Change prices and maybe rework.}

Per the variant rules notes, armour (including shields) grants an ``armour''
bonus, rather than granting an AC bonus. A character's AC bonus is half
their total ``armour bonus'' rounded down, and a character's DR bonus is
half their total ``armour bonus'' rounded up.

%
%
\section{Weapons}

Most weapons are \textbf{one tenth the cost} of the Core Rulebook and SRD
values. This is subject to adjudication by the DM on a case-by-case basis.

Weapons still cost more than the metal used to make them, by a fair bit.

Most ``exotic weapons'' are considered martial weapons for the cultures
that they are found in, and aren't found elsewhere. The DM will adjudicate
these on a case-by-case basis. About half of them are gimmicks from
high-fantasy campaigns and won't actually work as weapons.

Modified, and removed weapons are described below.

%
\subsection{Knife}

Per the Duanor guide, a knife's CRB/SRD stats would be as follows:

5 sp, 1d3 (P), crit 19-20/x2, range 10', 1/2 lb

Modified for World of Adventure rules, its stats are:

1 sp, 1d3 (P), crit 18-20/x2, range 10', 1/2 lb

(Critical threat range was increased per 2d10 rules, and cost is constrained
by the fact that half a pound of iron costs 0.5~sp.)
%

%
\subsection{Bows}

A shortbow can support two masterworking increments per category (as with
a one-handed melee weapon), and a longbow can support three increments per
category (as with a two-handed melee weapon).

In lieu of ``composite bows'' being a thing, all bows have a strength
rating. Standard bows have a strength rating of +0. Higher strength ratings
require masterworking increments (one increment for a +1 strength rating,
two for +2, etc).

If your strength bonus equals or exceeds a bow's strength rating, then per
point of the bow's rating you get +1 to damage and your range increment
increases by 10' (for a shortbow) or 20' (for a longbow).

If your strength bonus is lower than the bow's strength rating, you not only
get no bonus to damage or range increment, but additionally get a penalty.
For each point by which your strength bonus falls short, you have a -1 to
damage and your range increment decreases by 10' (for a shortbow) or 20'
(for a longbow).

Bows with negative strength bonus exist; these are poor-quality goods
(per the masterworking section).
%

%
\subsection{Removed Weapons}

The following weapons are removed:
\begin{itemize}
\item \textbf{Composite Bow} -- Replaced by the revised bow rules, per above.
\item \textbf{Nunchaku} -- You are more likely to hurt yourself than anyone
else. Use a flail instead.
\item \textbf{Double Axe} -- These tend to rotate when trying for the second
hit. Use a pole-axe or halberd if you want a double weapon (the second
strike is with the haft), or a greataxe if you want to chop things.
\item \textbf{Spiked Chain} -- This is is even less practical than
nunchaku.
\item \textbf{Gnomish Hooked Hammer} -- As with the double axe, these tend
to rotate when trying for the second hit. If you want to be able to choose
between bludgeoning and piercing, get a pick with a hammer-style head
(point on one side, flat on the other) instead of a double-ended weapon.
\item \textbf{Sword, Two-Bladed} -- As with the double axe, these tend
to rotate when trying for the second hit.
\item \textbf{Crossbow, Repeating} -- The only way to fire multiple shots
in quick succession is to have multiple bows stacked on top of each other,
which is not practical. The reload time for a normal crossbow is spent
drawing back the bow, not loading the bolts.
\end{itemize}

...Notably, the Dwarven Urgosh could actually work as a weapon since a
spear-head still works no matter how the shaft has rotated. Practicality
is questionable, but if you really want to use it, go for it.
%

%
%
\section{Adventuring}

\fixme{NYI}

%
%
\section{Alchemical}

Most alchemical items are \textbf{one tenth the cost} of the Core Rulebook
and SRD values. This is adjudicated by the DM on a case-by-case basis; ask
before making plans, as exceptions exist.

New and modified alchemical items are described below.

%
\subsection{Antitoxin}

Antitoxin gives a +5 to Fortitude saves vs poison or venom, decaying by -1
per hour after the antitoxin is administered (so drinking it right before
activities that risk exposure is optimal).

Antitoxins that target a specific class of poison (not venom) give a +10
bonus rather than +5, decaying by -1 per hour (as with general antitoxin).

Antitoxin costs 50~sp per dose.

With the ``spell-magic is even cheaper'' optional rule, antitoxin is made
obsolete by potions of Neutralize Poison.
%

%
\subsection{Antivenom}

Antivenom targets the venom of a specific class of creatures (as with
``Delay Specific Poison'' and ``Neutralize Specific Poison''). It gives
a +10 to fortitude saves vs that type of venom, decaying by -1 per hour
after administration (so drinking it right before activities that risk
exposure is ideal).

Common antivenoms cost 100~sp per dose. Uncommon ones cost 200~sp per dose.
Rare ones or ones that have to be made-to-order cost 400~sp per dose and
may require a sample of the venom in order to be made.

With the ``spell-magic is even cheaper'' optional rule, antivenom is made
obsolete by potions of Neutralize Poison.
%

%
\subsection{Poisons}

Poisons that are readily available will be sold in clearly marked vials in
a padded case or satchel. They will last a few months if they stay sealed,
and will typically degrade within hours after being applied to a weapon
(for safety). Using appropriate PPE and taking appropriate precautions when
applying them is still wise.

Most of these poisons are ``injury'' type (exposure is via a wound). Poisons
meant to be eaten/drunk (``ingested'') cost the same amount but are more
heavily restricted, and contact-poisons are rarely found outside of
alchemical laboratories.

One dose (one vial) can prepare one weapon or four pieces of ammunition.
Every hit with a weapon (whether or not it does damage) reduces the save DC
by 2 for subsequent attacks, to represent poison being wiped off. These
rules supersede the SRD's rules.

Poison is usually most effective against a broad class of creatures (mammals,
birds, reptiles, insects/arachnids, etc). It may be less effective or even
ineffective against other classes (at the DM's discretion).

Retail prices for poisons are computed as follows:
\begin{itemize}
\item Base price is 10~sp per dose.
\item First multiplier is x1 for DC~14, x2 for DC~16, x3 for DC~17, x4 for
DC~18.
\item Second multiplier is x1 for 4 rounds with 1 save, x2 for 6 rounds with
1 save, x3 for 6 rounds with 2 saves, x4 for 8 rounds with 2 saves.
\item Third multiplier is x1 for 1d2 ability damage, x2 for 1d3, x3 for 1d4,
and x4 for 1d3+1.
\item Damaging two ability scores instead of one increases the third
multiplier by one step.
\item Doing 1 wound point of damage in addition to ability damage increases
the third multiplier by one step.
\end{itemize}

The most expensive poisons tend to only be worth it under special
circumstances. Under most conditions, ``hire a hit squad with heavy
crossbows'' is cheaper.
%

%
%
\section{Career}


\fixme{NYI}
%
%
\section{Commodities}

About half of these are from the CRB; Duanor and Harbourton additions are
in bold.

\begin{tabular}{lll}\hline
\textbf{Item} & \textbf{Quantity} & \textbf{Cost (sp)} \\ \hline
wheat & 1 lb & 0.1 \\
flour & 1 lb & 0.2 \\
tobacco & 1 lb & 5 \\
cinnamon & 1 lb & 10 \\
ginger & 1 lb & 20 \\
pepper & 1 lb & 20 \\
salt & 1 lb & 50 \\
saffron & 1 lb & 150 \\
cloves & 1 lb & 150 \\
\hline
\end{tabular}

\begin{tabular}{lll}\hline
\textbf{Item} & \textbf{Quantity} & \textbf{Cost (sp)} \\ \hline
iron & 1 lb & 1 \\
\textbf{glass (poor)} & 1 lb & 1 \\
\textbf{glass (fine)} & 1 lb & 3 \\
\textbf{zinc} & 1 lb & 3 \\
\textbf{lead} & 1 lb & 3 \\
\textbf{pewter} & 1 lb & 3 \\
copper & 1 lb & 5 \\
\textbf{brass} & 1 lb & 5 \\
\textbf{bronze} & 1 lb & 6 \\
\textbf{tin} & 1 lb & 10 \\
silver & 1 lb & 50 \\
mercury & 1 lb & 150 \\
gold & 1 lb & 500 \\
\textbf{quicksilver} & 1 lb & 1250 \\
\textbf{mithral} & 1 lb & 2500 \\
\textbf{adamantine} & 1 lb & 5000 \\
\textbf{aurichalcum} & 1 lb & 10000 \\
\hline
\end{tabular}

\begin{tabular}{lll}\hline
\textbf{Item} & \textbf{Quantity} & \textbf{Cost (sp)} \\ \hline
canvas & 1 sq yd & 1 \\
\textbf{cotton} & bolt & 30 \\
\textbf{leather} & measure & 50 \\
\textbf{linen} & bolt & 25 \\
\textbf{silk} & bolt & 75 \\
\textbf{wool} & bolt & 40 \\
\hline
\end{tabular}

\begin{tabular}{lll}\hline
\textbf{Item} & \textbf{Quantity} & \textbf{Cost (sp)} \\ \hline
chicken & & 0.2 \\
goat & & 10 \\
sheep & & 20 \\
pig & & 30 \\
cow & & 100 \\
ox & & 150 \\
\hline
\end{tabular}

%
%
\section{Travel}

\fixme{NYI}

\iffalse
\begin{tabular}{lll}\hline
\textbf{Item} & \textbf{Quantity} & \textbf{Cost} \\ \hline
\hline
\end{tabular}
\fi


%
% This is the end of the file.
