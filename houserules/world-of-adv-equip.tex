% "World of Adventure" campaign guide - Equipment
% Written by Christopher Thomas.

\chapter{Equipment}
\label{sect-equip}

%
%
\section{Masterworking}

\textit{Adapted from Tony's Duanor campaign, with permission.}

Items may be of improved quality (better artistic quality, different
materials, or just better at their intended purpose). All of these are
covered by ``masterworking''. The rules here supersede the Core Rulebook
and SRD's rules for masterwork items.

Improvements are implemented using ``masterworking increments'' in any
of several categories. These are planned at the time an object is crafted,
and cannot be re-worked afterward.

%
\subsection{Cost and Crafting DC}

The cost of a category's masterworking is computed using an object's
``base cost'':
\begin{center}
$\mathbf{mw~cost} = \mathbf{increments}^2 \times \mathbf{base~cost} \div 2$
\end{center}

The amount by which a category's masterworking increases the crafting DC
is:
\begin{center}
$\mathbf{DC~increase} = \mathbf{increments}^2 \times 2$
\end{center}

For purposes of computing masterworking cost, the ``base cost'' is not
always the actual cost of a non-masterwork item. There are several special
cases:
\begin{itemize}
%
\item For items costing less than 100~sp, the ``base cost'' is higher than
the retail cost:\\
$\mathbf{base~cost} = sqrt( 100~\mathrm{sp} \times \mathbf{retail~cost} )$
%
\item For adding enchantability slots, the ``base cost'' is always at least
100~sp.
%
\item For adding ``artistic quality'', the masterworking cost is
\textbf{not divided by 2}. Drop the ``$\div 2$'' part of the equation.
%
\end{itemize}
%

%
\subsection{Special Materials}

Making objects out of gold or mithral or adamantine requires masterworking
(in the ``materials processing'' category), which adds masterworking cost
independently of the cost of the material itself.

\begin{itemize}
\item Bronze, iron, and (for jewellery) silver don't require masterworking.
\item Silver weapons and gold jewellery require one masterworking increment.
\item Mithral and aurichalcum require two masterworking increments.
\item Adamantine requires three masterworking increments.
\end{itemize}

For objects with significant weight (such as armour and weapons), the cost
of the material is computed based on that weight (remembering that mithral
is half the weight of iron). See the ``commodities'' section for material
prices.

For jewellery, the cost of an item made from more valuable material is
computed as a multiplier to the ``silver jewellery'' price:
\begin{itemize}
\item Bronze and brass are 1/4x.
\item Gold is 10x.
\item Mithral is 50x.
\item Adamantine is 100x.
\item Aurichalcum is 200x.
\end{itemize}

Decorations such as gemstones, enamel, and gilding/plating are not treated
explicitly, but are instead handwaved based on a combination of the
material and the artistic quality.

\subsubsection{Sale Price vs Crafting Price}

For computing the \textbf{crafting time} of an object made from precious
materials (or from cheaper materials, such as brass jewellery), calculate
the price as if it was made from default materials (but still using the
``materials processing'' masterwork increments).

For computing the \textbf{retail price} of the finished product:
\begin{itemize}
\item If there are ``artistic quality'' masterworking increments, the cost
of those increments is multiplied by the material price ratio.
\item If there are other masterworking increments, their cost is unchanged.
\item The base price of the object is adjusted per above (multiplied if it's
jewellery, or having the commodity material price added if it's a large
object).
\end{itemize}

The upshot is that fancy jewellery made out of gold sells for ten times
more than fancy jewellery made out of silver but doesn't take ten times as
long to make. See the ``fancy ring'' example below for a detailed
calculation.
%

%
\subsection{Artistic Quality}

This masterworking category makes objects look nicer or more impressive
without making them more functional.

A single masterworking improvement corresponds to the ``Superior'' quality
in the old Duanor rules, two improvements corresponds to ``Exceptional'',
three improvements to ``Outstanding'', and four to ``Legendary''.

Most people are capable of producing objects of Superior artistic quality,
but Exceptional quality or better requires either years of practice or the
Inspired Artist feat.

For adding ``artistic quality'' masterworking, the masterworking cost is
\textbf{not divided by 2}. Drop the ``$\div 2$'' part of the equation.
%

%
\subsection{Weapons and Armour}

Masterworking categories specific to weapons are as follows:
\begin{itemize}
\item ``Attack'' category -- Improve to-hit rolls (+1 per increment).
\item ``Damage'' category (melee only) --
Improve damage rolls (+1 per increment).
\item ``Speed'' category -- Improve initiative (+1 per increment).
\item ``Strength'' category (longbow and shortbow only) --
Improve damage and range, per the ``Bows'' section (+1 strength rating
per increment).
\end{itemize}

``Light'' weapons can support one increment per category, ``one-handed''
weapons can support two increments per category, and ``two-handed'' weapons
can support three increments per category. For purposes of these limits,
treat shortbows and light crossbows as ``one-handed'' and longbows and
heavy crossbows as ``two-handed''.

Masterworking categories specific to armour are as follows:
\begin{itemize}
\item ``Armour'' category -- Improve the armour rating (+1 per increment).
\item ``Mobility'' category -- Increase the maximum Dex bonus, reduce the
armour check penalty, and reduce spell failure chance. Each increment
improves the Dex limit and ACP by 1 and spell failure by 5\%.
\item Reduce weight by 10\% (per increment).
\end{itemize}

Light armour can support one increment per category, medium armour can
support two increments per category, and heavy armour can support three
increments per category.
%

%
\subsection{Tools}

Masterwork tools usually aren't bought individually, but are instead part
of the equipment kits described in the ``Career'' equipment section.

The +2 set of equipment includes a wide selection of tools of normal
quality and a small number of +1 masterwork tools. The +4 set of equipment
has a wide selection of +1 masterwork tools and several +2 masterwork tools.

The ``best I can get with minimal funds'' starting tools includes a few
vital tools of normal quality and a few more of poor quality.
%

%
\subsection{Enchantability}

Masterworking for enchantability involves making space for a runic substrate
(the magical equivalent of a printed circuit board), and for larger items
adding fine aurichalcum wire or plated traces to act as waveguides.
Supporting more magical effects requires a more complicated substrate, more
aurichalcum, and more gemstones to hold magical charge, and none of these
are cheap.

Each ``enchantability'' masterworking improvement adds two ``slots''.
Effects that are implemented take up slots:
\begin{itemize}
\item Spell effects take up one slot per spell level.
\item Enhancement bonuses to weapons and armour take up one slot per ``plus''.
\item Other bonuses to AC or armour take up one slot per ``plus''.
\item Bonuses to ability scores or saves take up one slot per \textbf{+2}
bonus or fraction thereof.
\item Bonuses to skills take up one slot per \textbf{+3} bonus or fraction
thereof.
\item Making an item function as a cleric's holy symbol or as a wizard's
bonded item takes up two slots.
\end{itemize}

Permanent magic items need this form of masterworking. For expendable items,
the cost is folded into the enchantment cost of the item.
%

%
\subsection{Poor Quality}

Items can be made with penalties rather than bonuses to some of their
traits. This makes them easier and faster to craft. Objects can be made
with 1 or 2 such traits, reducing the cost to 75\% or 50\% of the normal
cost, and reducing the crafting DC by 2 for each trait.

Typical traits are:
\begin{itemize}
\item -1 to a tool's skill checks.
\item -1 to a weapon's to-hit or damage rolls.
\item -1 to a piece of armour's armour rating.
\item Wears out quickly (see below).
\end{itemize}

This can't be combined with masterworking. It usually represents bad
apprentice work or someone trying to make a profit selling cheap goods to
people who can't afford anything better.

Regarding ``Wears Out Quickly'':
\begin{itemize}
\item Most equipment is assumed to be durable enough to last for many years
if properly cared for. For situations where this needs to be checked, roll
once per year and assume failure on 2 (1\%), 2-3 (3\%), or 2-4 (6\%),
depending on circumstances.
\item Equipment that wears out quickly is checked per month, rather than
per year.
\end{itemize}
%

%
\subsection{Examples}
%
\subsubsection{Standard Warding Amulet}

This is a silver medallion with enamelled artwork, on a sturdy steel chain.
It has two slots for enchantment: the ``Adventurer's Amulet'' version has
``Protection from Background Magic'' and ``Endure Elements'', while the
``Mind Ward'' version has ``Protection from Immaterial Influence'' and
``Protection from Mental Control''. Lots of other variants exist. It could
also be turned into a holy symbol or bonded object.

Retail cost of a silver medallion: 25~sp (DC 17).

``Base price'' for masterworking: 50~sp (the square root of 25~x~100), or
100~sp for enchantability.

Masterworking:
\begin{itemize}
\item +1 artistic quality (+50~sp, +2~DC)
\item +1 enchantability (+50~sp, +2~DC)
\end{itemize}

MSRP before actually adding the enchantments: 125~sp. Crafting DC: 21.
%
\subsubsection{Masterwork Reinforced Leather PPE}

This is reinforced leather PPE of the best available quality, with multiple
layers of boiled leather, metal shims at strategic locations, a padded
under-layer, and good tailoring to avoid hampering mobility. It's high-end
front-line police armour.

Retail cost of reinforced leather PPE: 25~sp (DC 17).

``Base price'' for masterworking: 50~sp (the square root of 25~x~100).

Masterworking:
\begin{itemize}
\item +1 armour (25~sp, +2~DC)
\item +1 mobility (25~sp, +2~DC)
\end{itemize}

MSRP: 75~sp. Crafting DC: 21.

Stats: armour +4, max dex +6, ACP 0, SF 10\%.

This is more difficult to make than a chain shirt, but offers the same
protection and has much better mobility.
%
\subsubsection{Fancy Ring}

This is a class ring or signet ring owned by someone who has money and
wants you to know it. The ring is white gold with elaborately patterned
accents in rose gold and yellow gold.

Retail cost of a silver ring: 75~sp (DC 19).

``Base price'' for masterworking: 87~sp (the square root of 75~x~100).

Made of gold (price multiplier x10), exceptional artistic quality.

Masterworking:
\begin{itemize}
\item +1 material (44~sp, +2 DC)
\item +2 art quality (348~sp, +8 DC)
\item +1 enchantability (+50~sp, +2 DC)
\end{itemize}

Price for calculating crafting time: 517~sp (75 ring, 348 art, 94 other).
Crafting DC: 31

MSRP before adding enchantments: 4324~sp (750 ring, 3480 art, 94 other)

Typical enchantments might be Comprehend Languages (always active), Detect
Poison (always active), and Arcane Mark (for adding invisible authentication
text when used as a signet ring or rolled seal).

%
%
\section{Armour}

%
\subsection{Armour as AC Plus DR}

Per the variant rules notes, armour (including shields) grants an ``armour''
bonus, rather than granting an AC bonus. A character's AC bonus is half
their total ``armour bonus'' rounded down, and a character's DR bonus is
half their total ``armour bonus'' rounded up.
%

%
\subsection{Donning Armour}

Times for donning armour in the Core Rulebook and SRD are unrealistic to
the point of being silly. Revised times are as follows:

\begin{tabular}{lll} \hline
\textbf{Type} & \textbf{Time Alone} & \textbf{With Help} \\ \hline
Light & 1d2 rounds & 1 round \\
Medium & 1d2+1 rounds & 1d2 rounds \\
Heavy & 1d3+2 rounds & 1d2+1 rounds \\
Shield & move action & n/a \\
\hline
\end{tabular}
%

%
\subsection{Armour Types and Prices}

Several armour types in the Core Rulebook either duplicate each other or
didn't actually exist, so these are being consolidated and revised. The
costs are also about \textbf{one tenth the cost} of the Core Rulebook and
SRD values (more or less).

Armour generally falls into four categories:
\begin{itemize}
\item Work clothes. This is mostly leather-based personal protective
equipment.
\item Chain armour. This may have partial coverage (chain shirt) or be more
complete.
\item Reinforced armour (for lack of a better term). This is cloth or
leather with metal components attached to it.
\item Plate armour. This may have partial coverage (breastplate) or be more
complete.
\end{itemize}

Adjusted prices and statistics for armour are given below. This could be
back-ported to be SRD-compatible by multiplying prices by 10.
\textbf{Statistics have been adjusted} slightly from the SRD versions.

\begin{tabular}{|l|ll|l|lll|}\hline
\textbf{Type} & \textbf{Cost (sp)} & \textbf{Weight (lbs)} &
\textbf{Bonus} & \textbf{Max Dex} & \textbf{ACP} & \textbf{Spell Fail} \\
\hline
Cloth PPE &
8 sp & 10 lbs & 1 & +8 & 0 & 5\% \\
Leather PPE &
15 sp & 15 lbs & 2 & +6 & 0 & 10\% \\
Reinforced Leather &
25 sp & 20 lbs & 3 & +5 & -1 & 15\% \\
Chain Shirt &
70 sp & 25 lbs & 4 & +4 & -2 & 20\% \\
\hline
Scale Mail &
50 sp & 30 lbs & 5 & +3 & -4 & 25\% \\
Chain Mail &
150 sp & 40 lbs & 6 & +2 & -5 & 30\% \\
Breastplate &
250 sp & 30 lbs & 6 & +3 & -4 & 25\% \\
\hline
Half-Plate &
600 sp & 40 lbs & 8 & +1 & -6 & 35\% \\
Full Plate &
1500 sp & 50 lbs & 9 & +0 & -7 & 40\% \\
\hline
Light Shield &
5 sp & 5 lbs & 1 & & -1 & 5\% \\
Heavy Shield &
10 sp & 10 lbs & 2 & & -2 & 15\% \\
\hline
\end{tabular}

Detailed descriptions are below. For any given type of armour, there are
many possible implementations and styles. Changes that would have been
reflected by different armour types in the Core Rulebook or SRD are instead
reflected by different degrees of masterworking within a category. One
category can encompass several types of armour.

\begin{itemize}
%
\item \textbf{Cloth PPE} is along the lines of heavy canvas work clothes.
Fancier versions may include padding or small amounts of leather.
%
\item \textbf{Leather PPE} is along the lines of welding (or smithing)
clothing. Fancier versions may include areas that are reinforced with
multiple layers, or incorporate boiled leather (hardened), or both.
%
\item \textbf{Reinforced Leather} is PPE that is expected to withstand
impact, sharp implements, or both. Modern examples are biking leathers
with polycarbonate plates, and steel-toed work boots. This typically has
padding underneath it.
%
\item \textbf{Chain Shirt} is chain mail with partial coverage. For
purposes of game rule abstraction, we're treating it mechanically the same
way we'd treat full-coverage armour.
%
\item \textbf{Scale Mail} is a catch-all category including scale, lamellar,
and brigandine armour. It is made from metal scales or small plates sewn
to a cloth or leather backing (or to each other, for lamellar armour). For
scale, the metal scales are on the outside; for brigandine, the metal plates
are on the inside. In both cases, padding is usually worn under it.
%
\item \textbf{Chain Mail} is chain armour with full coverage (torso, arms,
and legs).
%
\item \textbf{Breastplate} is plate armour covering the torso only. For
purposes of game rule abstraction, we're treating it mechanically the
same way we'd treat full-coverage armour.
%
\item \textbf{Half-Plate} is plate armour with partial protection for
the limbs. It typically includes a breastplate, vambraces, greaves, and
helmet. Fancier versions may include pauldrons and other components. A
real-world example would be Roman lorica segmentata.
%
\item \textbf{Full Plate} is plate armour with full coverage. Gaps may
have chain mail under them, or the armour may overlap in a way that leaves
no gaps.
%
\item \textbf{Shields} are made from laminated wooden planks covered with
leather, and sometimes reinforced with metal (typically at the edge). Per
the Core Rulebook, objects can be held in the shield hand with a light
shield but not with a heavy shield. The shield hand can't be used with
a weapon or to perform tasks requiring dexterity (drop the shield first).
%
\end{itemize}
%

%
\subsection{Removed Types}

The following types of armour from the Core Rulebook and SRD have been
removed:

\begin{itemize}
%
\item Hide armour has no role for PCs or for NPCs. Poor-quality or
improvised armour would be variants of leather instead.
%
\item A realistic version of splint mail would probably be high-end
reinforced leather under the house rules classification. The Core Rulebook
version serves little purpose (much worse mobility than a breastplate for
marginal AC improvement).
%
\item The Core Rulebook's ``banded mail'' was an RPG invention. The closest
real-world equivalent is Roman lorica segmentata, which is half-plate under
the house rules classification.
%
\item There was a real-world shield called a ``buckler'', but it was used
very differently than the Core Rulebook version. The Core Rulebook's
version was a way to get the benefit of a shield while still being able to
use that hand for weapons. Get armour with vambraces (and a better armour
rating) instead.
%
\end{itemize}
%

%
%
\section{Weapons}

Most weapons are \textbf{one tenth the cost} of the Core Rulebook and SRD
values. This is subject to adjudication by the DM on a case-by-case basis.

Weapons still cost more than the metal used to make them, by a fair bit.

Per the ``2d10'' house rule, \textbf{threat ranges are increased by 1}
over the SRD's values. Effects or feats that double threat ranges instead
increase them by 1.

Most ``exotic weapons'' are considered martial weapons for the cultures
that they are found in, and aren't found elsewhere. The DM will adjudicate
these on a case-by-case basis. About half of them are gimmicks from
high-fantasy campaigns and won't actually work as weapons.

Modified and removed weapons are described below.

%
\subsection{Knife}

Per the Duanor guide, a knife's CRB/SRD stats would be as follows:

5 sp, 1d3 (P), crit 19-20/x2, range 10', 1/2 lb

Modified for World of Adventure rules, its stats are:

1 sp, 1d3 (P), crit 18-20/x2, range 10', 1/2 lb

(Critical threat range was increased per 2d10 rules, and cost is constrained
by the fact that half a pound of iron costs 0.5~sp.)
%

%
\subsection{Bows}

A shortbow can support two masterworking increments per category (as with
a one-handed melee weapon), and a longbow can support three increments per
category (as with a two-handed melee weapon).

In lieu of ``composite bows'' being a thing, all bows have a strength
rating. Standard bows have a strength rating of +0. Higher strength ratings
require masterworking increments (one increment for a +1 strength rating,
two for +2, etc).

If your strength bonus equals or exceeds a bow's strength rating, then per
point of the bow's rating you get +1 to damage and your range increment
increases by 10' (for a shortbow) or 20' (for a longbow).

If your strength bonus is lower than the bow's strength rating, you not only
get no bonus to damage or range increment, but additionally get a penalty.
For each point by which your strength bonus falls short, you have a -1 to
damage and your range increment decreases by 10' (for a shortbow) or 20'
(for a longbow).

Bows with negative strength bonus exist; these are poor-quality goods
(per the masterworking section).
%

%
\subsection{Sap, Truncheon, and Other Police Weapons}

Nonlethal (or less-lethal) bludgeoning weapons are available, and are
widely used by law enforcement:

\begin{itemize}
%
\item The ``sap'' is a short weighted baton with a soft head. It costs 2~sp,
weighs 2~lbs, does 1d4 damage, and is otherwise identical to a light mace.
It is a Simple weapon. This replaces the Core Rulebook's ``sap''.
%
\item The ``truncheon'' is a longer weighted baton with padding. It costs
4~sp, weighs 4~lbs, does 1d6, and is otherwise identical to a heavy mace.
It is a Simple weapon.
%
\item The ``riot flail'' is a padded version of a heavy flail. It costs
10~sp, weighs 8~lbs, does 1d8, and is otherwise identical to a heavy flail.
It is a Martial weapon.
%
\item The ``man-catcher'' is a pole-arm intended to be used to pin a target
in place. It is clumsy to use as a weapon (treat it as doing 1d6~B of
lethal damage if used as such). Its intended function is to be used as a
Reach weapon to make Grapple checks (without provoking an attack of
opportunity). It grants a +2 bonus to CMB and CMD for purposes of initiating
and maintaining control of such a grapple. It costs 10~sp and weighs 12~lbs.
It is considered a Martial weapon for anyone who has police training, and an
Exotic weapon otherwise.
%
\end{itemize}
%

%
\subsection{Removed Weapons}

The following weapons are removed:
\begin{itemize}
\item \textbf{Composite Bow} -- Replaced by the revised bow rules, per above.
\item \textbf{Nunchaku} -- You are more likely to hurt yourself than anyone
else. Use a flail instead.
\item \textbf{Double Axe} -- These tend to rotate when trying for the second
hit. Use a pole-axe or halberd if you want a double weapon (the second
strike is with the haft), or a greataxe if you want to chop things.
\item \textbf{Spiked Chain} -- This is is even less practical than
nunchaku.
\item \textbf{Gnomish Hooked Hammer} -- As with the double axe, these tend
to rotate when trying for the second hit. If you want to be able to choose
between bludgeoning and piercing, get a pick with a hammer-style head
(point on one side, flat on the other) instead of a double-ended weapon.
\item \textbf{Sword, Two-Bladed} -- As with the double axe, these tend
to rotate when trying for the second hit.
\item \textbf{Crossbow, Repeating} -- The only way to fire multiple shots
in quick succession is to have multiple bows stacked on top of each other,
which is not practical. The reload time for a normal crossbow is spent
drawing back the bow, not loading the bolts.
\end{itemize}

...Notably, the Dwarven Urgosh could actually work as a weapon since a
spear-head still works no matter how the shaft has rotated. Practicality
is questionable, but if you really want to use it, go for it.
%

%
%
\section{Adventuring}

Most adventuring gear prices are the same as given in the Core Rulebook and
SRD. Exceptions are noted below. There may be additional exceptions; if a
CRB/SRD price looks strange, check with me for a house-ruling on it.

%
\subsection{Ink Pens and Ink}

An inkpen is 1~sp, per the Core Rulebook and SRD prices.

A vial of ink is 8~sp (not the SRD's 80~sp), because this is a literate
society with a mature alchemy industry.
%

%
\subsection{Mangifying Glass and Spyglass}

\begin{itemize}
\item A magnifying glass costs 100~sp.
\item A wearable magnifier costs 200~sp (similar to what you'd use with
electronics; one lens per eye).
\item A jeweller's loupe costs 200~sp (handheld unit with two lenses in a
short tube).
\item A spyglass costs 500~sp.
\item A binocular loupe costs 500~sp (wearable unit with two lenses per
eye, equivalent to a surgical loupe). This is rare (not many use-cases
where it's needed).
\item A pair of binoculars costs 1200~sp (one spyglass per eye with
adjustment mechanisms). This is rare (a spyglass does the job most of the
time).
\end{itemize}

A cantrip exists that does what a magnifying glass or wearable magnifier
would do.

%
\subsection{Rope and Chain}

You can get heavier rope (twice the test limit, half the length). This
has +2 to the break DC but -2 to the escape artist DC due to bulk. It will
also do a better job on Large creatures.

50' of ordinary hemp rope has 2 hit points, is broken with a DC~23 strength
check, costs 10~sp, and weighs 10~lbs.

50' of ordinary silk rope has 4 hit points, is broken with a DC~24 strength
check, costs 100~sp, and weighs 5~lbs.

You can get chain. A 10' length of standard chain costs 20~sp, weighs
10~lbs, has hardness 10, 5 hit points, and is broken with a DC~26 strength
check. You can get heavier chain that's half the length with +2 break DC,
as with rope.
%

%
\subsection{Tents}

Tents of various sizes are as follows:

\begin{tabular}{lll}\hline
\textbf{Size} & \textbf{Weight} & \textbf{Cost} \\ \hline
1-person & 12 lbs & 35 sp \\
2-person & 20 lbs & 60 sp \\
3-person & 30 lbs & 80 sp \\
4-person & 40 lbs & 100 sp \\
6-person & 60 lbs & 140 sp \\
\hline
\end{tabular}

Tents have enough space for a person's backpack and bedroll. To store more
gear, use a larger tent (with some ``occupants'' being cargo).
%

%
%
\section{Alchemical}

Most alchemical items are \textbf{one tenth the cost} of the Core Rulebook
and SRD values. This is adjudicated by the DM on a case-by-case basis; ask
before making plans, as exceptions exist.

New and modified alchemical items are described below.

%
\subsection{Antitoxin}

Antitoxin gives a +5 to Fortitude saves vs poison or venom, decaying by -1
per hour after the antitoxin is administered (so drinking it right before
activities that risk exposure is optimal).

Antitoxins that target a specific class of poison (not venom) give a +10
bonus rather than +5, decaying by -1 per hour (as with general antitoxin).

Antitoxin costs 50~sp per dose.

With the ``spell-magic is even cheaper'' optional rule, antitoxin is made
obsolete by potions of Neutralize Poison.
%

%
\subsection{Antivenom}

Antivenom targets the venom of a specific class of creatures (as with
``Delay Specific Poison'' and ``Neutralize Specific Poison''). It gives
a +10 to fortitude saves vs that type of venom, decaying by -1 per hour
after administration (so drinking it right before activities that risk
exposure is ideal).

Common antivenoms cost 100~sp per dose. Uncommon ones cost 200~sp per dose.
Rare ones or ones that have to be made-to-order cost 400~sp per dose and
may require a sample of the venom in order to be made.

With the ``spell-magic is even cheaper'' optional rule, antivenom is made
obsolete by potions of Neutralize Poison.
%

%
\subsection{Poisons}

Poisons that are readily available will be sold in clearly marked vials in
a padded case or satchel. They will last a few months if they stay sealed,
and will typically degrade within hours after being applied to a weapon
(for safety). Using appropriate PPE and taking appropriate precautions when
applying them is still wise.

Most of these poisons are ``injury'' type (exposure is via a wound). Poisons
meant to be eaten/drunk (``ingested'') cost the same amount but are more
heavily restricted, and contact-poisons are rarely found outside of
alchemical laboratories.

One dose (one vial) can prepare one weapon or four pieces of ammunition.
Every hit with a weapon (whether or not it does damage) reduces the save DC
by 2 for subsequent attacks, to represent poison being wiped off. These
rules supersede the SRD's rules.

Poison is usually most effective against a broad class of creatures (mammals,
birds, reptiles, insects/arachnids, etc). It may be less effective or even
ineffective against other classes (at the DM's discretion).

Retail prices for poisons are computed as follows:
\begin{itemize}
\item Base price is 10~sp per dose.
\item First multiplier is x1 for DC~14, x2 for DC~16, x3 for DC~17, x4 for
DC~18.
\item Second multiplier is x1 for 4 rounds with 1 save, x2 for 6 rounds with
1 save, x3 for 6 rounds with 2 saves, x4 for 8 rounds with 2 saves.
\item Third multiplier is x1 for 1d2 ability damage, x2 for 1d3, x3 for 1d4,
and x4 for 1d3+1.
\item Damaging two ability scores instead of one increases the third
multiplier by one step.
\item Doing 1 wound point of damage in addition to ability damage increases
the third multiplier by one step.
\end{itemize}

The most expensive poisons tend to only be worth it under special
circumstances. Under most conditions, ``hire a hit squad with heavy
crossbows'' is cheaper.
%

%
%
\section{Career}

Career-related equipment replaces the ``artisan's tools'' and ``alchemist's
laboratory'' entries in the Core Rulebook and SRD equipment lists.

\begin{itemize}
%
\item The bare minimum equipment needed to do a Crafting type job costs
50~sp. \textit{(Examples: ``artisan's tools'', ``musical instrument''.)}
For professions like alchemy or blacksmithing that need more expensive
tools, you are assumed to be working with a combination of second-hand
equipment and paying to use rented facilities.
%
\item Better equipment costs 500~sp. This gives you a +2 bonus to Craft or
Profession checks involving that equipment. This equipment takes up a fair
bit of space (a medium-sized workroom, at minimum).
%
\item The best equipment costs 3000~sp. This gives a +4 bonus to Craft or
Profession checks involving that equipment. This requires more space (a
large workroom and probably a bit of storage space as well).
%
\item Skill-boosting magic items are in roughly the same price bracket.
Most professionals invest in both.
%
\end{itemize}

%
%
\section{Commodities}

About half of these are from the CRB; Duanor and Harbourton additions are
in bold.

\begin{tabular}{lll}\hline
\textbf{Item} & \textbf{Quantity} & \textbf{Cost (sp)} \\ \hline
wheat & 1 lb & 0.1 \\
flour & 1 lb & 0.2 \\
tobacco & 1 lb & 5 \\
cinnamon & 1 lb & 10 \\
ginger & 1 lb & 20 \\
pepper & 1 lb & 20 \\
salt & 1 lb & 50 \\
saffron & 1 lb & 150 \\
cloves & 1 lb & 150 \\
\hline
\end{tabular}

\begin{tabular}{lll}\hline
\textbf{Item} & \textbf{Quantity} & \textbf{Cost (sp)} \\ \hline
iron & 1 lb & 1 \\
\textbf{glass (poor)} & 1 lb & 1 \\
\textbf{glass (fine)} & 1 lb & 3 \\
\textbf{zinc} & 1 lb & 3 \\
\textbf{lead} & 1 lb & 3 \\
\textbf{pewter} & 1 lb & 3 \\
copper & 1 lb & 5 \\
\textbf{brass} & 1 lb & 5 \\
\textbf{bronze} & 1 lb & 6 \\
\textbf{tin} & 1 lb & 10 \\
silver & 1 lb & 50 \\
mercury & 1 lb & 150 \\
gold & 1 lb & 500 \\
\textbf{quicksilver} & 1 lb & 1250 \\
\textbf{mithral} & 1 lb & 2500 \\
\textbf{adamantine} & 1 lb & 5000 \\
\textbf{aurichalcum} & 1 lb & 10000 \\
\hline
\end{tabular}

\begin{tabular}{lll}\hline
\textbf{Item} & \textbf{Quantity} & \textbf{Cost (sp)} \\ \hline
canvas & 1 sq yd & 1 \\
\textbf{cotton} & bolt & 30 \\
\textbf{leather} & measure & 50 \\
\textbf{linen} & bolt & 25 \\
\textbf{silk} & bolt & 75 \\
\textbf{wool} & bolt & 40 \\
\hline
\end{tabular}

\begin{tabular}{lll}\hline
\textbf{Item} & \textbf{Quantity} & \textbf{Cost (sp)} \\ \hline
chicken & & 0.2 \\
goat & & 10 \\
sheep & & 20 \\
pig & & 30 \\
cow & & 100 \\
ox & & 150 \\
\hline
\end{tabular}

%
%
\iffalse
%
\section{Travel}

\fixme{Travel NYI}

\iffalse
\begin{tabular}{lll}\hline
\textbf{Item} & \textbf{Quantity} & \textbf{Cost} \\ \hline
\hline
\end{tabular}
\fi
%
\fi

%
% This is the end of the file.
