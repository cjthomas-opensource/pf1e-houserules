% "World of Adventure" campaign guide - Overview
% Written by Christopher Thomas.

\chapter{Introduction}
\label{sect-over}

%
%
%
\section{What's Different?}
\label{sect-over-intro}

% FIXME - Point-form for now.

The most important points:
\begin{itemize}
%
\item This is \textit{realistic} high-fantasy. The economy works and magic
is used everywhere it makes economic sense to do so.
%
\item House rules have overhauled crafting/profession work, item prices, and
the way several spells work.
%
\end{itemize}

The setting:
\begin{itemize}
%
\item Magic is widely used where it makes economic sense to do so. It's a
skilled profession like any other.
%
\item Magical production of staple foods means that there's a large urban
population (instead of 90\% of the population farming, only half of it does).
%
\item Urbanization and the shift to a manufacturing economy means there's
no longer a feudal system. The default setting is the city-state of
Harbourton, governed by a council of guildmasters, with adjacent city-states
having their own governing structures (Ville Lumi\`{e}re having nobility,
Dwarven Canada having parliament, and the Orcish Federation being a
militocracy).
%
\item Adventuring is handled by guilds, which issue ``adventurer licenses''
along the same lines as the ``hero licenses'' from the 2001 ``Tick'' series.
Adventurer guilds work closely with and do contract work for their cities'
governments and law enforcement.
%
\item Gunpowder, steam power, and internal combusion don't work. Electricty
and magnetism exist but electric motors and generators don't work. Magic,
muscle power, wind power, and water power are the main drivers of industry.
The only way of getting mechanical power from point A to point B is an
aqueduct, and it's usually easier to move the factory to the water source.
Moving wood or charcoal to where industrial heat is needed is doable.
%
\item There are lots of still-wild areas to have adventures in. These have
strong background magic, which is about as healthy as strong background
radiation and which tends to generate wonders and monsters.
%
\end{itemize}

Variant rules with important effects:
\begin{itemize}
%
\item There is no teleport magic. There would be enough people able to cast
it that physical security would be nearly impossible.
%
\item There is no dimensional magic (Bag of Holding, Portable Hole, Rope
Trick, etc). For moving goods, a ``freight pallet'' version of Floating Disc
is used. For camping, people use tents (mundane or conjured).
%
\item The planes as described in the CRB do not exist. The only ones known
are the ``Material Realm'' (the every-day universe) and the ``Immaterial
Realm'' (containing thought, emotion, magic, and the gods). The realms
influence each other but physical travel between them is not possible (the
Immaterium isn't even a physical ``place'' as such).
%
\item Magic items are cheaper (to make them possible to afford at all). There
are several options for how \textit{much} cheaper they are, with different
consequences.
%
\item We're using 2d10 instead of 1d20. This makes extreme results much less
likely, and a +1 bonus makes a big difference when you're trying to do
something difficult.
%
\item We're using a variant of the ``wounds and vigor'' system. You gain
vigor points when you level up but not wound points. Some types of damage
(like fall damage) go straight to wounds.
%
\item We're using a variant of the ``armour as damage reduction'' rules. Most
forms of protection give a bonus to ``armour rating'', which gets split
between AC and DR.
%
\end{itemize}
%
\textit{I'd like to thank Anthony Nardelli for his work with the ``D\"{u}anor''
setting, which had similar variant rules and a working economy, directly
inspiring this setting.}

%
%
%
\section{Wealth and Resources}
\label{sect-over-econ}

% FIXME - Point-form for now.

\begin{itemize}
%
\item Large purchases tend to be counted in silver pieces. 1~sp is about
\$50 USD in 2025 funds. These are large silver coins. Smaller silver coins
and bronze coins of various sizes are used for day-to-day transactions.
Tokens issued by various vendors are usually tin (sometimes brass).
%
\item Income is a profession or craft check made weekly, with a
profession-specific multiplier. The check result is that week's income in
silver pieces. Income rolls also grant experience (half the check result
before the profession multiplier).
%
\item Different professions have different multipliers. Unskilled labour is
x1/2, skilled trades are x1, and professionals are x2. Hazard pay is
typically x2 on top of that. These are commodity rates; someone with skills
that can't easily be found can often get more.
%
\item Living expenses are abstracted as ``upkeep''. This can range from
6~sp/week for working-poor to 24~sp/week for upper-middle-class
professionals. It affects where you live and what you can buy before having
to track purchases explicitly. This is usually deducted monthly to make
paperwork easier.
%
\item A worker fresh out of apprenticeship who is not particularly talented
can average 15 on a profession roll. Someone with talent and Skill Focus can
manage 20 out of apprenticeship. Someone with talent, Skill Focus, good
tools, and experience can manage 25. Late-career professionals who invested
in equipment and accessories can manage an average roll of 30 (acting solo).
%
\item The default calendar has 5 weeks to the month and 40 working weeks per
year. Since lifestyle (and upkeep) tends to scale with income, most people
have 50--100~sp/year of disposable income, which tends to get spent during
the year on perks (fancy restaurant trips, going to a festival, buying a nice
piece of equipment, etc). Dedicated career-types may instead save it to
invest in job-related equipment.
%
\item About 1\% of the population are wealthy, with ten times the amount of
money flowing through their hands. 1\% of \textit{that} are extravagently
wealthy, with a hundred times that amount of money. Above that are rulers
and oligarchs.
%
\end{itemize}

%
%
%
\section{Industries}
\label{sect-over-demo}

% FIXME - Point-form for now.

The technology level is ``fantasy Renaissance'':
\begin{itemize}
%
\item Swords and bows are the dominant weapons of war (no gunpowder).
%
\item Water, animals, and people are the main source of motive power (no
engines). If magic is \textit{extremely} cheap, (see below) it takes over
as the main source of motive power.
%
\item Baseline food production is magical, via variants of ``Create Food
and Water'' with a number of cantrips used to re-flavour, re-texture, and
preserve conjured food. This allowed the transition from an agricultural
economy to a manufacturing economy.
%
\item Light sources are magical (typically using the ``light'' cantrip).
%
\item Freight transport is by carts and wagons drawn by horses that are
conjured using variants of the ``mount'' spell.
%
\item Factories exist, typically driven by water power (and built where
water power can be harnessed). Textile mills are a typical example. For
situations where less motive power is needed, a factory may use a ``prime
mover'' powered by the ``unseen oxen'' wheel-turning spell (either cast by
wizards or as a device, depending on how cheap magic is). Animal power
(such as via the ``mount'' spell) is sometimes used but needs more space.
%
\item Industries that either do not require or do not benefit from large
machinery use machines driven by treadle-power (as with pole lathes and
treadle-based sewing machines). These may also, where appropriate, be
driven by spells cast by the user (``apprentice's spinner'' or ``wizard's
wheel'').
%
\item Alchemy exists, and works much like the modern chemical industry.
Many specialized spells are used to augment non-magical tools. This is
similar to but explicitly not the same as real-world chemistry, to avoid
metagaming based on real-world knowledge.
%
\item Clockwork exists. Screws and bolts exist. Standardization of these
is iffy at best, and nonexistant at worst, so if you need repairs or
replacement parts you'll need to get them from the original manufacturer.
%
\item Several types of printing press exist. This is mainly limited by the
fact that paper production competes with several other uses for farmland,
and the fact that the master engravings (or dies, for typeset print) are
hand-made.
%
\item An entertainment industry exists. Spells for public address and for
sound manipulation are common and widely used, and illusion magic
supplements many live performances. Recording technology exists (using wax
cylinders or wax-coated discs), but nobody has managed to invent a good
high-fidelity duplication method. First-generation copies from a master
recording are expensive, with second-generation copies being less expensive
and sounding worse. Recordings wear out quickly with playback.
%
\item The existence of golems (magitek robots) is optional. They are
described in their own section.
%
\item There is a well-established medical profession. Treatment costs money,
because any given professional can only cast a few spells per day, and they
need to make a living at it. There are a large number of specialized spells
for healing, treating disease, and treating poison (lower-level than the
general-purpose spells for these things).
%
\item A contraception cantrip exists and is widely available and widely
used. This is a metagame choice: I wanted a setting with societies that can
exist for centuries without having a Malthusian collapse. Without
contraception population growth would instead be stabilized by war or famine.
%
\end{itemize}

%
% This is the end of the file.
