% "World of Adventure" campaign guide -
% Written by Christopher Thomas.

\chapter{Character Generation}
\label{sect-chargen}

%
%
\section{Concept}

Before starting to build your character, it's important to settle on a
\textit{concept}. You'll probably think about several different concepts
before picking one, and you'll probably fine-tune it during character
generation, but the concept is the starting point for everything else.

\begin{itemize}
%
\item Your character idea should be something that \textbf{you find
interesting}, so that you have fun playing it.
%
\item Your character idea should be something that \textbf{works well with
the group}. The other players' character concepts will usually suggest a
shared theme or set of shared goals (e.g. police team, pirates on the high
seas, etc).
%
\end{itemize}
%
Keep in mind that even with high-fantasy elements such as magic and dragons,
characters need to be able to function as real people. That dragon can hold
down a job (he runs the banking guild and his daughter is a police officer).

To help refine your concept, here are some questions you might ask about
your character:
\begin{itemize}
%
\item What do they want to do, or want to be, down the road?
%
\item How do they plan to get there?
%
\item What do they do for a living now, and what do they do for fun?
%
\item What are a few small things that they like or enjoy?
%
\item What are a few small things that they don't like or try to avoid?
%
\item How did they get to know the other player-characters? (Boring is fine
for this.)
%
\item How does joining the group bring them closer to their personal goals?
%
\end{itemize}

%
%
\section{Pathfinder Classes in a Civilized Setting}

In the ``World of Adventure'' setting, the player characters are probably
members of the Adventurers' Guild. ``Adventuring'' is what people do when
they think a normal job is too boring. This works the same way superhero
licenses from the first live-action ``Tick'' series did (2001).

Normal jobs that each of the Pathfinder Core Rulebook classes typically
have are as follows:

\begin{itemize}
%
\item \textbf{Barbarian} -- Talented but untrained warrior; common among
criminals.
%
\item \textbf{Bard} -- Entertainment industry and media.
%
\item \textbf{Cleric} -- Medical professionals.
%
\item \textbf{Druid} -- Forest rangers and rural veterinarians. Also have
spells to increase crop yields.
%
\item \textbf{Monk} -- (Not common; they'd either be performers or
adventurers.)
%
\item \textbf{Paladin} -- Police officer (the good kind).
%
\item \textbf{Ranger} -- Forest rangers and animal control.
%
\item \textbf{Rogue} -- Very common among criminals. Police have a few,
and a few are adventurers.
%
\item \textbf{Sorcerer} -- (Not common; usually from living in high magical
background.)
%
\item \textbf{Wizard} -- Alchemy (industrial chemistry), cosmetic illusion
industry, other specialists.
%
\item Most people in normal jobs have one of the NPC classes (such as
``Expert'') rather than a playable class.
%
\end{itemize}

About one person in a thousand decides to pursue an adventuring career; in
a city of half a million people, there are a few thousand. The authorities
fund and license this because it gives these people something to do instead
of causing trouble.

Adventuring activities are normally more-dangerous versions of normal
activities:

\begin{itemize}
%
\item Working with the police as a SWAT team.
%
\item Escorting civilians through extremely dangerous wilderness.
%
\item Animal control for very dangerous creatures.
%
\item Self-funded adventuring expeditions (looking for cool/valuable things
in dangerous high-background wilderness).
%
\end{itemize}

%
%
\section{Picking a Race and Generating Ability Scores}

Ability scores are bought using the point-buy system from the Pathfinder
Core Rulebook (pages 15--16). You get 20 points to spend, and the cost for
a given score is as follows:

\begin{tabular}{|l||l|l|l|l|l|l|}\hline
\textbf{Score} & 8 (-1) & 10 (--) & 12 (+1) & 14 (+2) & 16 (+3) & 18 (+4) \\
\hline
\textbf{Cost} & -2 & 0 & 2 & 5 & 10 & 17 \\
\hline
\end{tabular}

(Odd numbers omitted because the bonus or penalty only changes on even
values.)

Racial adjustments to ability scores happen after they're bought.

General guidelines:
\begin{itemize}
\item Don't have any ability lower than 8.
\item Don't have Cha or Con lower than 10.
\item Your most important ability score should be at least 16 (after racial
adjustment).
\end{itemize}

Your race will change your ability scores (usually boosting one or two and
dropping a third). Races from the Core Rulebook, from other Paizo sources,
and from third-party sources may be used with DM permission. Races used
in previous ``World of Adventure'' campaigns are summarized below.

All races age at the same speed as humans (no centuries-old elves).

As races are different species, they cannot normally interbreed (despite
lots of college kids trying). This means that half-elves aren't a thing,
and the CRB's ``half-orc'' race is retconned to be full orc (using the
half-orc racial attributes).

Racial attributes that are not consistent with a civilized setting (such
as dwarves having ``animosity'' towards orcs) are retconned out or replaced.

Weapon and armour familiarity depend on career, not race.

Typical races, from the Pathfinder Core Rulebook (modified), from the
Advanced Race Guide (modified), and from previous games are as follows.
Feel free to add more:

\begin{longtable}
%{l|l|l|l}
{p{0.1\columnwidth}|p{0.1\columnwidth}|p{0.35\columnwidth}|p{0.35\columnwidth}}
\hline
\textbf{Race} & \textbf{Stats} & \textbf{Skills} & \textbf{Other} \\ \hline
\endhead\endfoot
%
Dwarf & +2 Con\par +2 Wis\par -2 Cha &
+2 Appraise (metal/gems)\par +2 Perception (stonework) &
+2 save vs poison\par +2 save vs magic\par speed 20' indep. of load \\ \hline
%
Elf & +2 Dex\par +2 Int\par -2 Con &
+2 Perception\par +2 Spellcraft (identifying) &
low-light vision\par +5 save vs sleep\par +2 save vs enchantment \\ \hline
%
Gnome & +2 Con\par +2 Cha\par -2 Str &
+2 chosen Craft/Profession\par +2 Perception &
low-light vision\par small size\par +2 save vs illusion\par
+1 save DC for own illusion spells\par innate spells per CRB p23 \\ \hline
%
Goblin & +2 Int\par -2 Wis &
+2 chosen Craft/Profession\par
make a reflex save to escape disastrous skill botches without injury &
darkvision\par small size \\ \hline
%
Halfling & +2 Dex\par +2 Cha\par -2 Str &
+2 Acrobatics\par +2 Climb\par +2 Perception &
small size\par +2 save vs fear\par +1 all saves \\ \hline
%
Human & +2 any &
one extra skill point per level &
one extra feat at L1 \\ \hline
%
Kobold & +2 Dex\par -2 Str &
+2 chosen Craft/Profession &
small size\par fast (30' base speed)\par
darkvision\par natural armour 1 \\ \hline
%
Lizardfolk & +2 Str\par +2 Con\par -2 Wis &
+8 Swim &
bite (1d3)\par claws (1d4)\par natural armour 1 \\ \hline
%
Orc & +2 any &
+2 Intimidate &
darkvision\par fight one round below 0~hp \\ \hline
%
Rat-folk & +2 Dex\par &
+2 Perception &
small size\par fast (30' base speed)\par
low-light vision\par Scent ability \\ \hline
%
Tengu & +2 Dex\par +2 Wis\par -2 Con &
+4 Linguistics\par +2 Perception\par +2 Stealth &
bite (1d3 P)\par low-light vision \\ \hline
%
Wolf-folk & +2 Con &
+2 Survival &
low-light vision\par Scent ability \\ \hline
%
\end{longtable}

Ogres were an NPC race in at least one ``World of Adventure'' campaign, but
they are difficult to balance as player characters. It was also very
challenging to work out the logistics of a setting with size~S and size~L
people using the same buildings.

Goblins are re-imagined as being very much like Kerbals. The NPC versions
were ``born lucky'', giving them a skewed view of safe workplace practices
for things like alchemy.

%
% This is the end of the file.
