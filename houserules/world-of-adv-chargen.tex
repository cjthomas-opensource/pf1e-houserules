% "World of Adventure" campaign guide -
% Written by Christopher Thomas.

\chapter{Character Generation}
\label{sect-chargen}

%
%
\section{Concept}

Before starting to build your character, it's important to settle on a
\textit{concept}. You'll probably think about several different concepts
before picking one, and you'll probably fine-tune it during character
generation, but the concept is the starting point for everything else.

\begin{itemize}
%
\item Your character idea should be something that \textbf{you find
interesting}, so that you have fun playing it.
%
\item Your character idea should be something that \textbf{works well with
the group}. The other players' character concepts will usually suggest a
shared theme or set of shared goals (e.g. police team, pirates on the high
seas, etc).
%
\end{itemize}
%
Keep in mind that even with high-fantasy elements such as magic and dragons,
characters need to be able to function as real people. That dragon can hold
down a job (he runs the banking guild and his daughter is a police officer).

To help refine your concept, here are some questions you might ask about
your character:
\begin{itemize}
%
\item What do they want to do, or want to be, down the road?
%
\item How do they plan to get there?
%
\item What do they do for a living now, and what do they do for fun?
%
\item What are a few small things that they like or enjoy?
%
\item What are a few small things that they don't like or try to avoid?
%
\item How did they get to know the other player-characters? (Boring is fine
for this.)
%
\item How does joining the group bring them closer to their personal goals?
%
\end{itemize}

%
%
\section{Pathfinder Classes in a Civilized Setting}

In the ``World of Adventure'' setting, the player characters are probably
members of the Adventurers' Guild. ``Adventuring'' is what people do when
they think a normal job is too boring. This works the same way superhero
licenses from the first live-action ``Tick'' series did (2001).

Normal jobs that each of the Pathfinder Core Rulebook classes typically
have are as follows:

\begin{itemize}
%
\item \textbf{Barbarian} -- Talented but untrained warrior; common among
criminals.
%
\item \textbf{Bard} -- Entertainment industry and media.
%
\item \textbf{Cleric} -- Medical professionals.
%
\item \textbf{Druid} -- Forest rangers and rural veterinarians. Also have
spells to increase crop yields.
%
\item \textbf{Monk} -- (Not common; they'd either be performers or
adventurers.)
%
\item \textbf{Paladin} -- Police officer (the good kind).
%
\item \textbf{Ranger} -- Forest rangers and animal control.
%
\item \textbf{Rogue} -- Very common among criminals. Police have a few,
and a few are adventurers.
%
\item \textbf{Sorcerer} -- (Not common; usually from living in high magical
background.)
%
\item \textbf{Wizard} -- Alchemy (industrial chemistry), cosmetic illusion
industry, other specialists.
%
\item Most people in normal jobs have one of the NPC classes (such as
``Expert'') rather than a playable class.
%
\end{itemize}

About one person in a thousand decides to pursue an adventuring career; in
a city of half a million people, there are a few thousand. The authorities
fund and license this because it gives these people something to do instead
of causing trouble.

Adventuring activities are normally more-dangerous versions of normal
activities:

\begin{itemize}
%
\item Working with the police as a SWAT team.
%
\item Escorting civilians through extremely dangerous wilderness.
%
\item Animal control for very dangerous creatures.
%
\item Self-funded adventuring expeditions (looking for cool/valuable things
in dangerous high-background wilderness).
%
\end{itemize}

%
%
\section{Picking a Race and Generating Ability Scores}

Ability scores are bought using the point-buy system from the Pathfinder
Core Rulebook (pages 15--16). You get 20 points to spend, and the cost for
a given score is as follows:

\begin{tabular}{|l||l|l|l|l|l|l|}\hline
\textbf{Score} & 8 (-1) & 10 (--) & 12 (+1) & 14 (+2) & 16 (+3) & 18 (+4) \\
\hline
\textbf{Cost} & -2 & 0 & 2 & 5 & 10 & 17 \\
\hline
\end{tabular}

(Odd numbers omitted because the bonus or penalty only changes on even
values.)

Racial adjustments to ability scores happen after they're bought.

General guidelines:
\begin{itemize}
\item Don't have any ability lower than 8.
\item Don't have Cha or Con lower than 10.
\item Your most important ability score should be at least 16 (after racial
adjustment).
\end{itemize}

Your race will change your ability scores (usually boosting one or two and
dropping a third). Races from the Core Rulebook, from other Paizo sources,
and from third-party sources may be used with DM permission. Races used
in previous ``World of Adventure'' campaigns are summarized below.

All races age at the same speed as humans (no centuries-old elves).

As races are different species, they cannot normally interbreed (despite
lots of college kids trying). This means that half-elves aren't a thing,
and the CRB's ``half-orc'' race is retconned to be full orc (using the
half-orc racial attributes).

Racial attributes that are not consistent with a civilized setting (such
as dwarves having ``animosity'' towards orcs) are retconned out or replaced.

Weapon and armour familiarity depend on career, not race.

Typical races, from the Pathfinder Core Rulebook (modified), from the
Advanced Race Guide (modified), and from previous games are as follows.
Feel free to add more:

\begin{longtable}
%{l|l|l|l}
{p{0.1\columnwidth}|p{0.1\columnwidth}|p{0.35\columnwidth}|p{0.35\columnwidth}}
\hline
\textbf{Race} & \textbf{Stats} & \textbf{Skills} & \textbf{Other} \\ \hline
\endhead\endfoot
%
Dwarf & +2 Con\par +2 Wis\par -2 Cha &
+2 Appraise (metal/gems)\par +2 Perception (stonework) &
+2 save vs poison\par +2 save vs magic\par speed 20' indep. of load \\ \hline
%
Elf & +2 Dex\par +2 Int\par -2 Con &
+2 Perception\par +2 Spellcraft (identifying) &
low-light vision\par +5 save vs sleep\par +2 save vs enchantment \\ \hline
%
Gnome & +2 Con\par +2 Cha\par -2 Str &
+2 chosen Craft/Profession\par +2 Perception &
low-light vision\par small size\par +2 save vs illusion\par
+1 save DC for own illusion spells\par innate spells per CRB p23 \\ \hline
%
Goblin & +2 Int\par -2 Wis &
+2 chosen Craft/Profession\par
make a reflex save to escape disastrous skill botches without injury &
darkvision\par small size \\ \hline
%
Halfling & +2 Dex\par +2 Cha\par -2 Str &
+2 Acrobatics\par +2 Climb\par +2 Perception &
small size\par +2 save vs fear\par +1 all saves \\ \hline
%
Human & +2 any &
one extra skill point per level &
one extra feat at L1 \\ \hline
%
Kobold & +2 Dex\par -2 Str &
+2 chosen Craft/Profession &
small size\par fast (30' base speed)\par
darkvision\par natural armour 1 \\ \hline
%
Lizardfolk & +2 Str\par +2 Con\par -2 Wis &
+8 Swim &
bite (1d3 P)\par claws (1d4 S)\par natural armour 1 \\ \hline
%
Orc & +2 any &
+2 Intimidate &
darkvision\par fight one round below 0~hp \\ \hline
%
Rat-folk & +2 Dex\par +2 Int\par -2 Str &
+2 Perception\par +2 Craft (alchemy) &
small size\par low-light vision\par Scent ability \\ \hline
%
Tengu & +2 Dex\par +2 Wis\par -2 Con &
+4 Linguistics\par +2 Perception\par +2 Stealth &
bite (1d3 P)\par low-light vision \\ \hline
%
Wolf-folk & +2 Con &
+2 Survival &
low-light vision\par Scent ability \\ \hline
%
\end{longtable}

Ogres were an NPC race in at least one ``World of Adventure'' campaign, but
they are difficult to balance as player characters. It was also very
challenging to work out the logistics of a setting with size~S and size~L
people using the same buildings.

Goblins are re-imagined as being very much like Kerbals. The NPC versions
were ``born lucky'', giving them a skewed view of safe workplace practices
for things like alchemy.

%
%
\clearpage
\section{Finishing Touches}

%
\subsection{Languages}

Your character starts knowing the local language (usually named after the
region). This replaces the ``Common'' language in the Core Rules. Depending
on background, at the DM's discretion you may also know one additional
language (usually due to racial or cultural background), even if you don't
have the Int bonus normally needed for an additional language.

A character's Int bonus, as well as ranks put into Linguistics, gives them
additional ``language points'' to spend. These may be spent learning other
languages common to the region, or may be spent to ``master'' languages
already known.

Mastery in a language gives a +2 competence bonus to appropriate rolls
relating to that language (such as Profession: Writer checks and Diplomacy
checks with native speakers) and lets the character speak a second language
without an accent (or with a fabricated accent, with a Bluff check).

Each nation typically has a national language. If a nation has a majority
population of a given race (such as Ville Lumi\`{e}re being an elven-majority
nation), then members of that race in nearby countries are likely to have
the first country's language either as their mother tongue or as a second
language (such as the elves in Harbourton speaking Elvish in additon to
Harbourtonian).

Some races may also have their own language for practical reasons. In
Harbourton, most kobolds know Harbourtonian, but Draconic is much easier
for them to pronounce (which means it's usually what kobolds speak amongst
themselves, and it's usually a kobold's first language).

%
\subsection{Alignment}

I don't use alignment as-such; someone truly chaotic would not be able to
function in society, and someone truly evil would have to be very good at
masking to be able to hide for long.

That said, it's still important to know where your character stands on
moral issues, both for roleplay purposes and to help the DM run the game.
Your character will have interacted with many people over the years, and
those people will remember you and will remember how you act.

The following questions are good ones to think about:

\begin{itemize}
%
\item Does your character like having a structured environment, or hate
being told what to do, or not care?
%
\item Does your character go out of their way to help their friends? What
about helping strangers?
%
\item Would your character steal something from someone they know? What
about from a stranger? What if that stranger was a jerk or tried to fight
them?
%
\item What would it take to get your character angry enough to kill
someone? Would your character kill someone without being angry?
%
\item What are your character's ``red lines'' - things that they consider
important enough to hurt someone over, or to go to war over?
%
\end{itemize}

Sharing the answers to this sort of question with the DM and with other
players will help a lot.

%
\subsection{Starting Funds}

Your character started earning a living at around age 16 (earlier than in
our world), or perhaps 18 if they are in a career that required extensive
training. Prior to that (from about age 10 or 12), they were an apprentice
training under established professionals, or (for highly trained careers)
doing the equivalent of attending university or vocational training for
part of that time.

A character at the beginning of their career (in a career with an x1 income
multiplier) will start with about 200~sp worth of equipment, representing
what the've managed to accumulate while getting established. At the DM's
discretion, character background or higher or lower income multipliers may
modify this amount.

\fixme{Sanity check this against revised equipment costs for each class.}

%
\subsection{Older Characters}

Characters working ordinary careers at x1 income who are making an effort
to save money to invest in equipment will have an additional 200~sp of
equipment per year spent working. Higher or lower income multipliers will
raise or lower this amount.

% Derivation notes: 20 sp/week gross, at lower-middle-class upkeep (12 sp/k).
% This gives a maximum of 320 sp/year; saving two thirds of that at most.

Character level progression from work experience is as follows (skilled NPC
average):

\begin{tabular}{|l||c|c|c|c|c|c|c|}\hline
\textbf{Level} & 1 & 2 & 3 & 4 & 5 & 6 & 7 \\ \hline
\textbf{Age} & 16 & 22 & 30 & 40 & 50 & 65 & 90 \\ \hline
\end{tabular}

Player-characters are exceptional: They've been picked up by the ``Winds
of Destiny'', growing in skill far faster than most other people (due to
adventure XP awards). So, when starting at higher level, the DM might let
you start younger than most NPCs of that level would be.

%
% This is the end of the file.
